
%----------------------------------------------------------------------------------------
%	PROBLEM 1
%----------------------------------------------------------------------------------------

% To have just one problem per page, simply put a \clearpage after each problem

\begin{homeworkProblem}
Une approche très répandue dans l’étude de la propagation des logiciels malveillants
consiste à développer un modèle déterministe basé sur les concepts de compartiments et
de règles [2]. Les compartiments servent à diviser la population étudiée en différentes
classes et les règles à définir les conditions de transition entre chacune des classes.
\begin{homeworkSection}{1.1}
En vous basant sur l’article « Optimising Networks Against Malware » [3], quel
modèle comportemental (SI, SIS, SIR) s’appliquerait et pourquoi? Justifiez votre réponse
en expliquant quel modèle s’applique et pourquoi les autres modèles ne s’appliquent pas.

\problemAnswer{ % Answer
Dans un premier temps
}
\end{homeworkSection}
\end{homeworkProblem}

% %----------------------------------------------------------------------------------------
% %	PROBLEM 2
% %----------------------------------------------------------------------------------------

\begin{homeworkProblem}
% Question
Lors de la question précédente, vous avez développé un modèle théorique basé sur un
système d’équations différentielles. Heureusement, il existe une solution analytique à ce
système afin de représenter le nombre de machines infectées en fonction du temps :
\begin{equation}
I(t)=\frac{I_0 N}{(N-I_0)e^{-\lambda t}+I_0}
\label{moneq}
\end{equation}

% %--------------------------------------------

% \begin{homeworkSection}{(a)} % Section within problem
% \lipsum[4]\vspace{10pt} % Question

% %\problemAnswer{ % Answer
% %\lipsum[5]
% %}
%\end{homeworkSection}

% %--------------------------------------------

% \begin{homeworkSection}{(b)} % Section within problem
% %\problemAnswer{ % Answer
% %\lipsum[6]
% %}
% \end{homeworkSection}

% %--------------------------------------------
\end{homeworkProblem}