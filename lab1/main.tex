%%%%%%%%%%%%%%%%%%%%%%%%%%%%%%%%%%%%%%%%%
% Structured General Purpose Assignment
% LaTeX Template
%
% This template has been downloaded from:
% http://www.latextemplates.com
%
% Original author:
% Ted Pavlic (http://www.tedpavlic.com)
%
% Note:
% The \lipsum[#] commands throughout this template generate dummy text
% to fill the template out. These commands should all be removed when 
% writing assignment content.
%
%%%%%%%%%%%%%%%%%%%%%%%%%%%%%%%%%%%%%%%%%

%----------------------------------------------------------------------------------------
%	PACKAGES AND OTHER DOCUMENT CONFIGURATIONS
%----------------------------------------------------------------------------------------

\documentclass{article}

\usepackage[T1]{fontenc}
\usepackage[utf8]{inputenc}
\usepackage[francais]{babel}
\usepackage{fancyhdr} % Required for custom headers
\usepackage{lastpage} % Required to determine the last page for the footer
\usepackage{extramarks} % Required for headers and footers
\usepackage{graphicx} % Required to insert images
\usepackage{lipsum} % Used for inserting dummy 'Lorem ipsum' text into the template
\usepackage{amsmath,amsfonts,amssymb}


% Margins
\topmargin=-0.45in
\evensidemargin=0in
\oddsidemargin=0in
\textwidth=6.5in
\textheight=9.0in
\headsep=0.25in 

\linespread{1.1} % Line spacing

% Set up the header and footer
\pagestyle{fancy}
\lhead{\hmwkAuthorName} % Top left header
\chead{\hmwkClass: \hmwkTitle} % Top center header
\rhead{\firstxmark} % Top right header
\lfoot{\lastxmark} % Bottom left footer
\cfoot{} % Bottom center footer
\rfoot{Page\ \thepage\ de\ \pageref{LastPage}} % Bottom right footer
\renewcommand\headrulewidth{0.4pt} % Size of the header rule
\renewcommand\footrulewidth{0.4pt} % Size of the footer rule

\setlength\parindent{0pt} % Removes all indentation from paragraphs

%----------------------------------------------------------------------------------------
%	DOCUMENT STRUCTURE COMMANDS
%	Skip this unless you know what you're doing
%----------------------------------------------------------------------------------------

% Header and footer for when a page split occurs within a problem environment
\newcommand{\enterProblemHeader}[1]{
\nobreak\extramarks{#1}{#1 continue sur la page suivante\ldots}\nobreak
\nobreak\extramarks{#1 (continued)}{\textit{#1} continue sur la page suivante\ldots}\nobreak
}

% Header and footer for when a page split occurs between problem environments
\newcommand{\exitProblemHeader}[1]{
\nobreak\extramarks{#1 (continued)}{#1 continue sur la page suivante\ldots}\nobreak
\nobreak\extramarks{#1}{}\nobreak
}

\setcounter{secnumdepth}{0} % Removes default section numbers
\newcounter{homeworkProblemCounter} % Creates a counter to keep track of the number of problems

\newcommand{\homeworkProblemName}{}
\newenvironment{homeworkProblem}[1][Question \arabic{homeworkProblemCounter}]{ % Makes a new environment called homeworkProblem which takes 1 argument (custom name) but the default is "Problem #"
\stepcounter{homeworkProblemCounter} % Increase counter for number of problems
\renewcommand{\homeworkProblemName}{#1} % Assign \homeworkProblemName the name of the problem
\section{\homeworkProblemName} % Make a section in the document with the custom problem count
\enterProblemHeader{\homeworkProblemName} % Header and footer within the environment
}{
\exitProblemHeader{\homeworkProblemName} % Header and footer after the environment
}

\newcommand{\problemAnswer}[1]{ % Defines the problem answer command with the content as the only argument
\noindent\framebox[\columnwidth][c]{\begin{minipage}{0.98\columnwidth}#1\end{minipage}} % Makes the box around the problem answer and puts the content inside
}

\newcommand{\homeworkSectionName}{}
\newenvironment{homeworkSection}[1]{ % New environment for sections within homework problems, takes 1 argument - the name of the section
\renewcommand{\homeworkSectionName}{#1} % Assign \homeworkSectionName to the name of the section from the environment argument
\subsection{\homeworkSectionName} % Make a subsection with the custom name of the subsection
\enterProblemHeader{\homeworkProblemName\ [\homeworkSectionName]} % Header and footer within the environment
}{
\enterProblemHeader{\homeworkProblemName} % Header and footer after the environment
}
   
%----------------------------------------------------------------------------------------
%	NAME AND CLASS SECTION
%----------------------------------------------------------------------------------------

\newcommand{\hmwkTitle}{Rapport laboratoire 1} % Assignment title
\newcommand{\hmwkDueDate}{Mardi,\ 19\ Janvier\ 2016} % Due date
\newcommand{\hmwkClass}{INF-6422} % Course/class
\newcommand{\hmwkClassTime}{} % Class/lecture time
\newcommand{\hmwkClassInstructor}{Francois Labrèche} % Teacher/lecturer
\newcommand{\hmwkAuthorName}{Thomas Luinaud, Paul Berthier} % Your name

%----------------------------------------------------------------------------------------
%	TITLE PAGE
%----------------------------------------------------------------------------------------

\title{
\textmd{\textbf{\hmwkClass:\ \hmwkTitle}}\\
\normalsize\vspace{0.1in}\small{Rendu\ le\ \hmwkDueDate}\\
\vspace{0.1in}\large{à \textit{\hmwkClassInstructor\ \hmwkClassTime}}\\
\vspace{1in}
\includegraphics[width=200pt]{pictures/logo-poly.png}
\vspace{2in}
}

\author{\textbf{\hmwkAuthorName}}
\date{} % Insert date here if you want it to appear below your name

%----------------------------------------------------------------------------------------

\begin{document}

	\maketitle
	
	%----------------------------------------------------------------------------------------
	%	TABLE OF CONTENTS
	%----------------------------------------------------------------------------------------
	
	%\setcounter{tocdepth}{1} % Uncomment this line if you don't want subsections listed in the ToC
	
	\newpage
	\tableofcontents
	\newpage
	
	\begin{homeworkProblem}[Mise en contexte]
	
	
		\begin{homeworkSection}{1.1}
		
			L’épidémiologie pourrait être définie comme l’étude des rapports existant entre les
			maladies ou tout autre phénomène biologique, et divers facteurs susceptibles d’exercer
			une influence sur leur fréquence, distribution et évolution. Entre d’autres mots,
			l’épidémiologie s’intéresse aux facteurs qui influencent la santé des populations. Plus
			particulièrement, l’épidémiologie s’intéresse, entre autre, à étudier la dynamique de
			propagation des maladies infectieuses afin d’établir des stratégies de prévention et
			d’intervention permettant de diminuer l’impact sur la santé publique. À cet effet, la
			modélisation mathématique s’est révélée particulièrement intéressante afin de simuler des
			scénarios épidémiologiques, d’évaluer les risques associés et de quantifier l’efficacité et
			l’impact de différentes méthodes d’intervention et de prévention. Plusieurs approches
			peuvent être retenues, telles que les simulations numériques, les modèles déterministes ou
			encore les modèles stochastiques. Chaque approche présente des avantages et des
			inconvénients. Il convient donc de choisir la méthode la plus appropriée en fonction des
			questions de recherche auxquelles vous souhaitez répondre.
			
		\end{homeworkSection}
		
		\begin{homeworkSection}{1.2}
		
			Appliquée à la sécurité informatique, l’épidémiologie pourrait être vue comme l’étude
			des différents facteurs qui influencent la fréquence, la distribution et l’évolution des
			logiciels malveillants. Plus particulièrement, l’approche épidémiologique a inspiré de
			nombreux travaux de recherche portant sur l’étude de la propagation des logiciels
			malveillants. Le présent laboratoire vous permettra de vous familiariser avec certaines
			approches mathématiques fréquemment utilisées afin de modéliser la propagation de
			logiciels malveillants au sein d’un réseau.
		
		\end{homeworkSection}
	
	\end{homeworkProblem}
	
	
%----------------------------------------------------------------------------------------
%	PROBLEM 1
%----------------------------------------------------------------------------------------

% To have just one problem per page, simply put a \clearpage after each problem

\begin{homeworkProblem}
Une approche très répandue dans l’étude de la propagation des logiciels malveillants
consiste à développer un modèle déterministe basé sur les concepts de compartiments et
de règles [2]. Les compartiments servent à diviser la population étudiée en différentes
classes et les règles à définir les conditions de transition entre chacune des classes.
\begin{homeworkSection}{1.1}
En vous basant sur l’article « Optimising Networks Against Malware » [3], quel
modèle comportemental (SI, SIS, SIR) s’appliquerait et pourquoi? Justifiez votre réponse
en expliquant quel modèle s’applique et pourquoi les autres modèles ne s’appliquent pas.

\problemAnswer{ % Answer
Dans un premier temps
}
\end{homeworkSection}
\end{homeworkProblem}

% %----------------------------------------------------------------------------------------
% %	PROBLEM 2
% %----------------------------------------------------------------------------------------

\begin{homeworkProblem}
% Question
Lors de la question précédente, vous avez développé un modèle théorique basé sur un
système d’équations différentielles. Heureusement, il existe une solution analytique à ce
système afin de représenter le nombre de machines infectées en fonction du temps :
\begin{equation}
I(t)=\frac{I_0 N}{(N-I_0)e^{-\lambda t}+I_0}
\label{moneq}
\end{equation}

% %--------------------------------------------

% \begin{homeworkSection}{(a)} % Section within problem
% \lipsum[4]\vspace{10pt} % Question

% %\problemAnswer{ % Answer
% %\lipsum[5]
% %}
%\end{homeworkSection}

% %--------------------------------------------

% \begin{homeworkSection}{(b)} % Section within problem
% %\problemAnswer{ % Answer
% %\lipsum[6]
% %}
% \end{homeworkSection}

% %--------------------------------------------
\end{homeworkProblem}
	% %----------------------------------------------------------------------------------------
% %	PROBLEM 3
% %----------------------------------------------------------------------------------------


\begin{homeworkProblem}

\begin{homeworkSection}{3.1}

\problemAnswer{ % Answer
}
\end{homeworkSection}
\end{homeworkProblem}

% %----------------------------------------------------------------------------------------
% %	PROBLEM 2
% %----------------------------------------------------------------------------------------

\begin{homeworkProblem}
% Question


\end{homeworkProblem}


\end{document}
