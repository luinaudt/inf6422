	\begin{homeworkProblem}[Méthode statistique]
	
			Une approche souvent utilisée revient à filtrer les messages électroniques sur la base de
			leur contenu (content-based filtering). Un exemple classique consiste à filtrer les
			messages en fonction de la fréquence d’apparition de certains mots. Utilisez le fichier
			spambase afin d’appliquer une méthode statistique qui vous permettra de classifier les
			messages en deux catégories, soit spam (1) ou non spam (0).
			Compte tenu de la nature binomiale (0/1) de la variable dépendante, la régression
			logistique  peut être utilisée comme méthode de classification. Votre variable
			dépendante (0/1) représente la catégorie associée (spam ou non spam) et les variables
			indépendantes représentent la fréquence d’apparition de certains mots.	
	
		\begin{homeworkSection}{2.1}

			Effectuez une régression logistique en utilisant l’ensemble des variables (57)
			contenues dans le fichier spambase. Divisez les données afin d’en utiliser 66\% pour la
			phase d’apprentissage de votre modèle. À quoi servira l’autre 33\% ?
			Donnez pour chaque variable indépendante son coefficient ainsi que son « odd ratio »
			associé. Quelle est la signification de ces deux valeurs?\\

			\problemAnswer{
			
			Les premiers 66\% servent à établir notre modèle, et les derniers 33\% servent à évaluer sa performance (on doit appliquer notre modèle à des données qui n'ont pas été utilisées durant la phase d'apprentissage, sinon quoi les résultats seraient biaisés).
			
			Les coefficients et les "odd ratio" de chacune des variables sont sur le tableau \ref{coefs}.\\
			En ce qui concerne les coefficients, on effectue la somme des coefficients de chaque variable multiplié par le nombre d’occurrences du mot dans le mail. Si le résultat est positif, on considère que le mail est du spam. Ainsi, plus le coefficient est élevé, plus la variable correspondante est un indicateur de spam. Au contraire, un coefficient négatif est un indicateur que le mail est sûrement légitime.\\
			Le "odd ratio"est calculé comme $\frac{p}{1-p}$, avec p la probabilité que le mail dans lequel se trouve la variable correspondante soit du spam. Ainsi, un grand odd ratio est un indicateur de spam, et un petit odd ratio (inférieur à 1) indique au contraire que le mail est sûrement légitime.
			
			}
			
			\begin{table}[htbp]
			\footnotesize
			\begin{center}
			\begin{tabular}{|l|r|r|}
				\hline
				\textbf{Variable} & \multicolumn{1}{l|}{\textbf{Coefficient}} & \multicolumn{1}{l|}{\textbf{Odd ratio}} \\ \hline
				word\_freq\_make & -0.3895 & 0.6774 \\ \hline
				word\_freq\_address & -0.1458 & 0.8644 \\ \hline
				word\_freq\_all & 0.1141 & 1.1209 \\ \hline
				word\_freq\_3d & 2.2514 & 9.5012 \\ \hline
				word\_freq\_our & 0.5624 & 1.7549 \\ \hline
				word\_freq\_over & 0.883 & 2.418 \\ \hline
				word\_freq\_remove & 2.2785 & 9.7622 \\ \hline
				word\_freq\_internet & 0.5696 & 1.7676 \\ \hline
				word\_freq\_order & 0.7343 & 2.084 \\ \hline
				word\_freq\_mail & 0.1275 & 1.1359 \\ \hline
				word\_freq\_receive & -0.2557 & 0.7744 \\ \hline
				word\_freq\_will & -0.1383 & 0.8708 \\ \hline
				word\_freq\_people & -0.0796 & 0.9235 \\ \hline
				word\_freq\_report & 0.1447 & 1.1556 \\ \hline
				word\_freq\_addresses & 1.2362 & 3.4424 \\ \hline
				word\_freq\_free & 1.0386 & 2.8252 \\ \hline
				word\_freq\_business & 0.9599 & 2.6113 \\ \hline
				word\_freq\_email & 0.1203 & 1.1279 \\ \hline
				word\_freq\_you & 0.0813 & 1.0847 \\ \hline
				word\_freq\_credit & 1.0474 & 2.8503 \\ \hline
				word\_freq\_your & 0.2419 & 1.2737 \\ \hline
				word\_freq\_font & 0.2013 & 1.223 \\ \hline
				word\_freq\_000 & 2.2452 & 9.4426 \\ \hline
				word\_freq\_money & 0.4264 & 1.5317 \\ \hline
				word\_freq\_hp & -1.9204 & 0.1465 \\ \hline
				word\_freq\_hpl & -1.0402 & 0.3534 \\ \hline
				word\_freq\_george & -11.7673 & 0 \\ \hline
				word\_freq\_650 & 0.4454 & 1.5612 \\ \hline
				word\_freq\_lab & -2.4864 & 0.0832 \\ \hline
				word\_freq\_labs & -0.3299 & 0.719 \\ \hline
				word\_freq\_telnet & -0.1702 & 0.8435 \\ \hline
				word\_freq\_857 & 2.5488 & 12.7917 \\ \hline
				word\_freq\_data & -0.7383 & 0.4779 \\ \hline
				word\_freq\_415 & 0.6679 & 1.9501 \\ \hline
				word\_freq\_85 & -2.0554 & 0.128 \\ \hline
				word\_freq\_technology & 0.9237 & 2.5186 \\ \hline
				word\_freq\_1999 & 0.0465 & 1.0476 \\ \hline
				word\_freq\_parts & -0.5968 & 0.5506 \\ \hline
				word\_freq\_pm & -0.865 & 0.421 \\ \hline
				word\_freq\_direct & -0.3046 & 0.7374 \\ \hline
				word\_freq\_cs & -45.0481 & 0 \\ \hline
				word\_freq\_meeting & -2.6887 & 0.068 \\ \hline
				word\_freq\_original & -1.2471 & 0.2873 \\ \hline
				word\_freq\_project & -1.5732 & 0.2074 \\ \hline
				word\_freq\_re & -0.7923 & 0.4528 \\ \hline
				word\_freq\_edu & -1.4592 & 0.2324 \\ \hline
				word\_freq\_table & -2.3259 & 0.0977 \\ \hline
				word\_freq\_conference & -4.0156 & 0.018 \\ \hline
				char\_freq\_; & -1.2911 & 0.275 \\ \hline
				char\_freq\_( & -0.1881 & 0.8285 \\ \hline
				char\_freq\_[ & -0.6574 & 0.5182 \\ \hline
				char\_freq\_! & 0.3472 & 1.4151 \\ \hline
				char\_freq\_\$ & 5.336 & 207.683 \\ \hline
				char\_freq\_\# & 2.4032 & 11.0581 \\ \hline
				capital\_run\_length\_average & 0.012 & 1.0121 \\ \hline
				capital\_run\_length\_longest & 0.0091 & 1.0092 \\ \hline
				capital\_run\_length\_total & 0.0008 & 1.0008 \\ \hline
			\end{tabular}
			\end{center}
			\caption{Coefficients et "odd ratio" de chacune des variables}
			\label{coefs}
			\end{table}
			
		\end{homeworkSection}
		
		\begin{homeworkSection}{2.2}
			
			Évaluez et discutez des performances de votre modèle en termes de : taux de vrai
			positif, taux de faux positif, précision, sensibilité (recall), « F-measure » et l’aire sous la
			courbe (ROC area). Expliquez la signification de chacune de ces mesures.
			Donnez la matrice de confusion de votre modèle et expliquez ce qu’elle représente. 
	
			\problemAnswer{
			
			}
			
		\end{homeworkSection}
			
		\begin{homeworkSection}{2.3}
			
			Donnez un exemple de contre-mesure de type Tokenization attack qu’un spammeur
			pourrait facilement utiliser afin de contourner un filtre basé uniquement sur la fréquence
			d’apparition de certains mots. Votre méthode ne doit pas modifier la signification du
			message et ne doit pas ajouter de nouveaux mots. Appliquez votre méthode au message
			suivant:\\

			\textit{DEAR RECEIVER,\\ \\	
			You have just received a Taliban virus. Since we are not so technologically
			advanced in Afghanistan, this is a MANUAL virus. Please click on
			this link (http://clickme.com) to delete all the files on your hard disk
			yourself and send this mail to everyone you know.
			Thank you very much for helping us.\\	\\
			-Taliban hacker.}

			\problemAnswer{
			
			}
					
		\end{homeworkSection}
		
			
	\end{homeworkProblem}