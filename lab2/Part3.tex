	\begin{homeworkProblem}[Apprentissage automatique]
	
	
		Une autre catégorie de méthodes qui a fait ses preuves en détection de spam consiste à
		s’inspirer de l’intelligence artificielle et de recourir à des techniques dites d’apprentissage
		automatique (machine learning). Ces méthodes ont l’avantage de s’adapter et
		d’apprendre pour continuellement améliorer leur performance. 

	
		\begin{homeworkSection}{3.1}

			Les méthodes d’apprentissage automatique peuvent être divisées en plusieurs
			catégories. Parmi les plus courantes, nous retrouvons l’apprentissage non-supervisé,
			semi-supervisé, et supervisé. Expliquez les caractéristiques de chacune de ces méthodes.

			\problemAnswer{
			Pour l'analyse et la catégorisation de grande quantité de données, les réseaux de neurones sont de plus en plus utilisés. 
			Pour pouvoir utiliser un réseau de neurones il faut cependant l'entrainer.
			Dans cette partie, nous allons présenter 3 méthodes pour entrainer un réseau de neurones.
			
			
			\paragraph{L'apprentissage non-supervisé.}
			Dans cette méthode, c'est l'algorithme qui effectue le tri de classe.
			Pour cela, il les traites comme un ensemble de variable aléatoire.
			L'apprentissage non-supervisé a donc l'avantage de ne pas nécessiter un expert.
			\cite{apprentissage_np}

			
			\paragraph{L'apprentissage supervisé.}
			L'apprentissage supervisé consiste à donné un jeu de donnée étiqueté au réseau de neurones.
			Contrairement à l'apprentissage non-supervisé, cette méthode nécessite d'étiqueter tout le jeu de donnée par un expert.
			Une fois le réseau "entrainé", il devrait être capable de catégorisé une entrée automatiquement.
			
			\paragraph{L'apprentissage semi-supervisé.}
			Cette méthode regroupe l'apprentissage supervisé et non-supervisé.			
			Dans l'apprentissage semi-supervisé, on utilise deux set de données, un étiqueté et un non étiqueté.
			Celui-ci permet de trier plus facilement des grands ensembles de données.
			\cite{apprentissage_sp}
			
			}
			
		\end{homeworkSection}
		
		\begin{homeworkSection}{3.2}

			La classification naïve bayésienne (naive bayes classifier) est un exemple de
			méthode qui peut être utilisée afin de résoudre des problèmes de classification par
			apprentissage supervisé. Appliquez cette méthode au fichier spambase afin de filtrer les
			messages en fonction des 57 variables continues. Utilisez 66\% des données pour la phase
			d’apprentissage.
			
			Évaluez et discutez des performances de votre modèle en termes de : taux de vrai positif,
			taux de faux positif, précision, sensibilité (recall), « F-measure » et l’aire sous la courbe
			(ROC area). Donnez la matrice de confusion.
	
			\problemAnswer{
			
			}
			
		\end{homeworkSection}
			
		\begin{homeworkSection}{3.3}

			Une autre méthode de classification, les forêts d’arbres décisionnels (random forest), consiste à
			effectuer un apprentissage sur plusieurs arbres de décisions. Appliquez
			cette méthode au fichier spambase et utilisez les 57 variables continues. Utilisez 66\% des
			données pour la phase d’apprentissage.
			Évaluez et discutez des performances de votre modèle en termes de : taux de vrai positif,
			taux de faux positif, précision, sensibilité (recall), « F-measure » et l’aire sous la courbe
			(ROC area). Donnez la matrice de confusion.

			\problemAnswer{
			
			}
					
		\end{homeworkSection}
		
		\begin{homeworkSection}{3.4}

			Donnez un exemple de contre-mesure de type Statistical attack qu’un spammeur
			pourrait utiliser afin d’échapper à un filtre basé sur la fréquence des mots en utilisant une
			méthode d’apprentissage automatique. Votre exemple de contre-mesure peut impliquer de
			modifier le contenu du message. Proposez une solution qui permettrait de contrecarrer
			cette contre-mesure.

			\problemAnswer{
			
			}
					
		\end{homeworkSection}
		
			
	\end{homeworkProblem}