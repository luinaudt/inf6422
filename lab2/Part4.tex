	\begin{homeworkProblem}[Performance et optimisation]
	

		\begin{homeworkSection}{4.1}

			Comparez et discutez, en termes de performance, les résultats que vous avez obtenus
			pour les différentes méthodes utilisées (régression logistique, classification naïve
			bayésienne, forêts d’arbres décisionnels). Selon vos résultats, quelles méthodes semblent
			donner les meilleures performances pour le jeu de données spambase?\\

			\problemAnswer{

				On peut tout de suite éliminer la méthode naïve bayésienne, qui apporte beaucoup trop de faux positifs.
				Les méthodes de la régression logistique et de la forêt d'arbres décisionnels ont des résultats semblables plutôt corrects, avec un taux de faux positifs inférieur à 5\% et un taux de vrais négatifs inférieur à 10\%. Cependant, la méthode de la forêt d'arbres décisionnels a l'avantage de se baser sur l'apprentissage automatique, ce qui veut dire que ses résultats s'amélioreront au fil du temps contrairement à celle de la régression logistique. C'est donc la solution la plus performante parmi les trois proposées, sur le jeu de donnés \textit{spambase}.  			
			
			}
			
		\end{homeworkSection}
		
		\begin{homeworkSection}{4.2}
			
			Nommez un avantage et un inconvénient d’utiliser des filtres basés sur
			l’apprentissage machine supervisé. Comparez avec l’apprentissage machine non
			supervisé en donnant un avantage et un inconvénient dans un contexte de détection de
			spam. Selon vous, est-ce qu’une méthode semble plus appropriée? Justifiez votre réponse
			et n’oubliez pas de donner vos références.\\
	
			\problemAnswer{
			Les méthodes de classification supervisée ont l'inconvénient de nécessiter un expert capable de définir les caractéristiques du jeu de données en entrée \cite{apprentissage_s}.
			Cependant, elles permettent des précisions plus élevées, mais elles posent le problème d'être limitées dans le temps.
			En effet, il faut ré-entrainer le réseau de neurones, si l'on veut ajouter la détection de nouveau spam.
			
			Les méthodes d'apprentissage non supervisé proposent des résultat de classification moins performants.
			Elles ont toutefois l'avantage de ne pas nécessiter d'intervention humaine, et le besoin de ré-entrainer le réseau de neurones est plus faible \cite{spring:classComp}.
			
			
			Finalement, une bonne approche serait d'utiliser le réseau de neurones avec un apprentissage par renforcement \cite{wiki-renforcement}, on pourrait ainsi demander à l'utilisateur de détecter les nouveaux spam.
			}
			
		\end{homeworkSection}
			
		\begin{homeworkSection}{4.3}

			Le jeu de données spambase est principalement basé sur la fréquence d’apparition de
			certains mots et sur la présence de lettres majuscule. Donnez au moins deux autres
			exemples de caractéristiques (features) qui pourraient être prises en compte afin
			d’améliorer la performance des modèles de classification que vous avez développés.
			Donnez vos références. \\

			\problemAnswer{
			
			En se basant sur l'article \cite{survey} donné sur Moodle, on peut améliorer la performance de notre algorithme de classification en y intégrant les caractéristiques suivantes :
			
			\begin{itemize}
			\item Des informations du header comme le champs "FROM", la date, la taille du message ou encore le nombre de pièces jointes
			\item Les images présentent dans le mail, car du texte peut y être écrit
			\item L'ordre d'apparition des mots
			\end{itemize}
			
			
			
			}
					
		\end{homeworkSection}
		
			
	\end{homeworkProblem}