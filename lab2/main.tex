%%%%%%%%%%%%%%%%%%%%%%%%%%%%%%%%%%%%%%%%%
% Structured General Purpose Assignment
% LaTeX Template
%
% This template has been downloaded from:
% http://www.latextemplates.com
%
% Original author:
% Ted Pavlic (http://www.tedpavlic.com)
%
% Note:
% The \lipsum[#] commands throughout this template generate dummy text
% to fill the template out. These commands should all be removed when 
% writing assignment content.
%
%%%%%%%%%%%%%%%%%%%%%%%%%%%%%%%%%%%%%%%%%

%----------------------------------------------------------------------------------------
%	PACKAGES AND OTHER DOCUMENT CONFIGURATIONS
%----------------------------------------------------------------------------------------

\documentclass{article}

\usepackage[T1]{fontenc}
\usepackage[utf8]{inputenc}
\usepackage[francais]{babel}
\usepackage{fancyhdr} % Required for custom headers
\usepackage{lastpage} % Required to determine the last page for the footer
\usepackage{extramarks} % Required for headers and footers
\usepackage{graphicx} % Required to insert images
\usepackage{lipsum} % Used for inserting dummy 'Lorem ipsum' text into the template
\usepackage{amsmath,amsfonts,amssymb}
\usepackage{pgfplots}

% Margins
\topmargin=-0.45in
\evensidemargin=0in
\oddsidemargin=0in
\textwidth=6.5in
\textheight=9.0in
\headsep=0.25in 

\linespread{1.1} % Line spacing

% Set up the header and footer
\pagestyle{fancy}
\lhead{\hmwkAuthorName} % Top left header
\rhead{\hmwkClass: \hmwkTitle} % Top center header
\lfoot{\lastxmark} % Bottom left footer
\cfoot{} % Bottom center footer
\rfoot{Page\ \thepage\ de\ \pageref{LastPage}} % Bottom right footer
\renewcommand\headrulewidth{0.4pt} % Size of the header rule
\renewcommand\footrulewidth{0.4pt} % Size of the footer rule

\setlength\parindent{0pt} % Removes all indentation from paragraphs

%----------------------------------------------------------------------------------------
%	DOCUMENT STRUCTURE COMMANDS
%	Skip this unless you know what you're doing
%----------------------------------------------------------------------------------------

% Header and footer for when a page split occurs within a problem environment
\newcommand{\enterProblemHeader}[1]{
\nobreak\extramarks{#1}{#1 continue sur la page suivante\ldots}\nobreak
\nobreak\extramarks{#1 (continued)}{\textit{#1} continue sur la page suivante\ldots}\nobreak
}

% Header and footer for when a page split occurs between problem environments
\newcommand{\exitProblemHeader}[1]{
\nobreak\extramarks{#1 (continued)}{#1 continue sur la page suivante\ldots}\nobreak
\nobreak\extramarks{#1}{}\nobreak
}

\setcounter{secnumdepth}{0} % Removes default section numbers
\newcounter{homeworkProblemCounter} % Creates a counter to keep track of the number of problems

\newcommand{\homeworkProblemName}{}
\newenvironment{homeworkProblem}[1][]{ % Makes a new environment called homeworkProblem which takes 1 argument (custom name) but the default is "Problem #"
\stepcounter{homeworkProblemCounter} % Increase counter for number of problems
\renewcommand{\homeworkProblemName}{\arabic{homeworkProblemCounter}. #1} % Assign \homeworkProblemName the name of the problem
\section{\homeworkProblemName} % Make a section in the document with the custom problem count
\enterProblemHeader{\homeworkProblemName} % Header and footer within the environment
}{
\exitProblemHeader{\homeworkProblemName} % Header and footer after the environment
}

\newcommand{\problemAnswer}[1]{ % Defines the problem answer command with the content as the only argument
\noindent\framebox[\columnwidth][c]{\begin{minipage}{0.98\columnwidth}#1\end{minipage}} % Makes the box around the problem answer and puts the content inside
}

\newcommand{\homeworkSectionName}{}
\newenvironment{homeworkSection}[1]{ % New environment for sections within homework problems, takes 1 argument - the name of the section
\renewcommand{\homeworkSectionName}{#1} % Assign \homeworkSectionName to the name of the section from the environment argument
\subsection{\homeworkSectionName} % Make a subsection with the custom name of the subsection
\enterProblemHeader{\homeworkProblemName\ [\homeworkSectionName]} % Header and footer within the environment
}{
\enterProblemHeader{\homeworkProblemName} % Header and footer after the environment
}
   
%----------------------------------------------------------------------------------------
%	NAME AND CLASS SECTION
%----------------------------------------------------------------------------------------

\newcommand{\hmwkTitle}{Rapport laboratoire 2} % Assignment title
\newcommand{\hmwkDueDate}{Mardi,\ 16\ février\ 2016} % Due date
\newcommand{\hmwkClass}{INF-6422} % Course/class
\newcommand{\hmwkClassTime}{} % Class/lecture time
\newcommand{\hmwkClassInstructor}{François Labrèche} % Teacher/lecturer
\newcommand{\hmwkAuthorName}{Thomas LUINAUD (1702271), Paul BERTHIER (1757237)} % Your name

%----------------------------------------------------------------------------------------
%	TITLE PAGE
%----------------------------------------------------------------------------------------

\title{
\textmd{\textbf{\hmwkClass:\ \hmwkTitle}}\\
\normalsize\vspace{0.1in}\small{Rendu\ le\ \hmwkDueDate}\\
\vspace{0.1in}\large{à \textit{\hmwkClassInstructor\ \hmwkClassTime}}\\
\vspace{1in}
\includegraphics[width=200pt]{pictures/logo-poly.png}
\vspace{2in}
}

\author{\textbf{\hmwkAuthorName}}
\date{} % Insert date here if you want it to appear below your name

%----------------------------------------------------------------------------------------

\begin{document}

	\maketitle
	
	%----------------------------------------------------------------------------------------
	%	TABLE OF CONTENTS
	%----------------------------------------------------------------------------------------
	
	%\setcounter{tocdepth}{1} % Uncomment this line if you don't want subsections listed in the ToC
	
	\newpage
	\tableofcontents
	\newpage
	
	\begin{homeworkProblem}[Mise en contexte]
	
	
		\begin{homeworkSection}{1.1}
		
			Le spam, ou encore courriel indésirable, peut être défini comme étant une
			communication électronique non sollicitée. Selon le rapport Q1-2014 de Kaspersky,
			le spam représenterait aujourd’hui plus de la moitié du trafic électronique. Souvent utilisé
			pour des fins commerciales, le spam peut aussi être utilisé pour fins d’escroquerie ou
			prendre la forme d’hameçonnage (phishing en anglais) afin de tromper le destinataire
			dans le but d’obtenir des informations personnelles. 
			
		\end{homeworkSection}
		
		\begin{homeworkSection}{1.2}
		
			Bien que le coût d’envoi d’un message électronique puisse être négligeable pour les
			spammeurs, celui associé à sa réception peut causer des coûts non négligeables tant aux
			destinataires qu’aux prestataires de services compte tenu du volume élevé d’envoi.
			Plusieurs méthodes de détection ont par conséquent été développées afin de filtrer les
			messages indésirables.
			Une première catégorie de filtres consiste à bloquer les messages sur la base d’une liste.
			Cette méthode peut elle-même être divisée en plusieurs techniques, soit le recours aux
			listes noires (blacklisting), aux listes blanches (whitelisting), aux listes grises
			(greylisting), etc.
			Une autre approche qui a démontré son efficacité consiste à filtrer les messages sur la
			base de leur contenu en utilisant des méthodes d’apprentissage automatique. Le présent
			travail pratique vous permettra de vous familiariser avec certaines méthodes
			d’apprentissage automatique et d’en évaluer la performance dans un contexte de
			détection de spam. 
					
		\end{homeworkSection}
	
	\end{homeworkProblem}
	
		\begin{homeworkProblem}[Méthode statistique]
	
			Une approche souvent utilisée revient à filtrer les messages électroniques sur la base de
			leur contenu (content-based filtering). Un exemple classique consiste à filtrer les
			messages en fonction de la fréquence d’apparition de certains mots. Utilisez le fichier
			spambase afin d’appliquer une méthode statistique qui vous permettra de classifier les
			messages en deux catégories, soit spam (1) ou non spam (0).
			Compte tenu de la nature binomiale (0/1) de la variable dépendante, la régression
			logistique  peut être utilisée comme méthode de classification. Votre variable
			dépendante (0/1) représente la catégorie associée (spam ou non spam) et les variables
			indépendantes représentent la fréquence d’apparition de certains mots.	
	
		\begin{homeworkSection}{2.1}

			Effectuez une régression logistique en utilisant l’ensemble des variables (57)
			contenues dans le fichier spambase. Divisez les données afin d’en utiliser 66\% pour la
			phase d’apprentissage de votre modèle. À quoi servira l’autre 33\% ?
			Donnez pour chaque variable indépendante son coefficient ainsi que son « odd ratio »
			associé. Quelle est la signification de ces deux valeurs?\\

			\problemAnswer{
			
			Les premiers 66\% servent à établir notre modèle, et les derniers 33\% servent à évaluer sa performance (on doit appliquer notre modèle à des données qui n'ont pas été utilisées durant la phase d'apprentissage, sinon quoi les résultats seraient biaisés).
			
			Les coefficients et les "odd ratio" de chacune des variables sont sur le tableau \ref{coefs}.\\
			En ce qui concerne les coefficients, on effectue la somme des coefficients de chaque variable multiplié par le nombre d’occurrences du mot dans le mail. Si le résultat est positif, on considère que le mail est du spam. Ainsi, plus le coefficient est élevé, plus la variable correspondante est un indicateur de spam. Au contraire, un coefficient négatif est un indicateur que le mail est sûrement légitime.\\
			Le "odd ratio"est calculé comme $\frac{p}{1-p}$, avec p la probabilité que le mail dans lequel se trouve la variable correspondante soit du spam. Ainsi, un grand odd ratio est un indicateur de spam, et un petit odd ratio (inférieur à 1) indique au contraire que le mail est sûrement légitime.
			
			}
			
			\begin{table}[htbp]
			\footnotesize
			\begin{center}
			\begin{tabular}{|l|r|r|}
				\hline
				\textbf{Variable} & \multicolumn{1}{l|}{\textbf{Coefficient}} & \multicolumn{1}{l|}{\textbf{Odd ratio}} \\ \hline
				word\_freq\_make & -0.3895 & 0.6774 \\ \hline
				word\_freq\_address & -0.1458 & 0.8644 \\ \hline
				word\_freq\_all & 0.1141 & 1.1209 \\ \hline
				word\_freq\_3d & 2.2514 & 9.5012 \\ \hline
				word\_freq\_our & 0.5624 & 1.7549 \\ \hline
				word\_freq\_over & 0.883 & 2.418 \\ \hline
				word\_freq\_remove & 2.2785 & 9.7622 \\ \hline
				word\_freq\_internet & 0.5696 & 1.7676 \\ \hline
				word\_freq\_order & 0.7343 & 2.084 \\ \hline
				word\_freq\_mail & 0.1275 & 1.1359 \\ \hline
				word\_freq\_receive & -0.2557 & 0.7744 \\ \hline
				word\_freq\_will & -0.1383 & 0.8708 \\ \hline
				word\_freq\_people & -0.0796 & 0.9235 \\ \hline
				word\_freq\_report & 0.1447 & 1.1556 \\ \hline
				word\_freq\_addresses & 1.2362 & 3.4424 \\ \hline
				word\_freq\_free & 1.0386 & 2.8252 \\ \hline
				word\_freq\_business & 0.9599 & 2.6113 \\ \hline
				word\_freq\_email & 0.1203 & 1.1279 \\ \hline
				word\_freq\_you & 0.0813 & 1.0847 \\ \hline
				word\_freq\_credit & 1.0474 & 2.8503 \\ \hline
				word\_freq\_your & 0.2419 & 1.2737 \\ \hline
				word\_freq\_font & 0.2013 & 1.223 \\ \hline
				word\_freq\_000 & 2.2452 & 9.4426 \\ \hline
				word\_freq\_money & 0.4264 & 1.5317 \\ \hline
				word\_freq\_hp & -1.9204 & 0.1465 \\ \hline
				word\_freq\_hpl & -1.0402 & 0.3534 \\ \hline
				word\_freq\_george & -11.7673 & 0 \\ \hline
				word\_freq\_650 & 0.4454 & 1.5612 \\ \hline
				word\_freq\_lab & -2.4864 & 0.0832 \\ \hline
				word\_freq\_labs & -0.3299 & 0.719 \\ \hline
				word\_freq\_telnet & -0.1702 & 0.8435 \\ \hline
				word\_freq\_857 & 2.5488 & 12.7917 \\ \hline
				word\_freq\_data & -0.7383 & 0.4779 \\ \hline
				word\_freq\_415 & 0.6679 & 1.9501 \\ \hline
				word\_freq\_85 & -2.0554 & 0.128 \\ \hline
				word\_freq\_technology & 0.9237 & 2.5186 \\ \hline
				word\_freq\_1999 & 0.0465 & 1.0476 \\ \hline
				word\_freq\_parts & -0.5968 & 0.5506 \\ \hline
				word\_freq\_pm & -0.865 & 0.421 \\ \hline
				word\_freq\_direct & -0.3046 & 0.7374 \\ \hline
				word\_freq\_cs & -45.0481 & 0 \\ \hline
				word\_freq\_meeting & -2.6887 & 0.068 \\ \hline
				word\_freq\_original & -1.2471 & 0.2873 \\ \hline
				word\_freq\_project & -1.5732 & 0.2074 \\ \hline
				word\_freq\_re & -0.7923 & 0.4528 \\ \hline
				word\_freq\_edu & -1.4592 & 0.2324 \\ \hline
				word\_freq\_table & -2.3259 & 0.0977 \\ \hline
				word\_freq\_conference & -4.0156 & 0.018 \\ \hline
				char\_freq\_; & -1.2911 & 0.275 \\ \hline
				char\_freq\_( & -0.1881 & 0.8285 \\ \hline
				char\_freq\_[ & -0.6574 & 0.5182 \\ \hline
				char\_freq\_! & 0.3472 & 1.4151 \\ \hline
				char\_freq\_\$ & 5.336 & 207.683 \\ \hline
				char\_freq\_\# & 2.4032 & 11.0581 \\ \hline
				capital\_run\_length\_average & 0.012 & 1.0121 \\ \hline
				capital\_run\_length\_longest & 0.0091 & 1.0092 \\ \hline
				capital\_run\_length\_total & 0.0008 & 1.0008 \\ \hline
			\end{tabular}
			\end{center}
			\caption{Coefficients et "odd ratio" de chacune des variables}
			\label{coefs}
			\end{table}
			
		\end{homeworkSection}
		
		\begin{homeworkSection}{2.2}
			
			Évaluez et discutez des performances de votre modèle en termes de : taux de vrai
			positif, taux de faux positif, précision, sensibilité (recall), « F-measure » et l’aire sous la
			courbe (ROC area). Expliquez la signification de chacune de ces mesures.
			Donnez la matrice de confusion de votre modèle et expliquez ce qu’elle représente.\\
	
			\problemAnswer{
			
			Weka nous donne les performances de notre modèle à l'aide du tableau de la figure \ref{accuracy} 1.
			Voici la signification des différentes valeurs fournies :
			
			\begin{itemize}
			\item \textbf{Taux de vrai positif} : C'est le taux de messages correctement classifiés comme spam.
			\item \textbf{Taux de faux positif} : C'est le taux de messages classifiés comme spam alors qu'ils étaient en réalité légitimes.
			\item \textbf{Précision} : C'est le rapport entre le nombre de messages correctement classifiés comme spam sur le nombre total de messages classifiés comme spam.
			\item \textbf{Sensibilité} :C'est le rapport entre le nombre de messages correctement classifiés comme spam sur le nombre de messages étant réellement du spam. C'est l'équivalent du taux de vrai positif.
			\item \textbf{F-measure} : C'est une formule donnant un critère de performance sous forme d'une moyenne pondérés de la précision et de la sensibilité. On la calcule de la façon suivante : $ F = 2* \frac{precision*sensibilite}{precision + sensibilite}$.
			\item \textbf{Aire sous la courbe} : La courbe représente le taux de vrais-positifs en fonction du taux de faux-positifs. Soient un message de spam et un message légitime choisis aléatoirement, et la question "Lequel de ces deux messages est du spam ?".  L'aire sous la courbe représente alors la probabilité que notre classificateur réponde correctement à cette question.\\
			\end{itemize}
			
			Les résultats obtenus sont les suivantes :
			
			\begin{itemize}
			\item \textbf{Taux de faux positif} : 0,046
			\item \textbf{Taux de vrai positif} : 0,891
			\item \textbf{Précision} : 0,926
			\item \textbf{Sensibilité} : 0,891
			\item \textbf{<<F-Measure>>} : 0,908
			\item \textbf{Aire sous la courbe} : 0,971
			\end{itemize}
			
			Pour un système classificateur de mails, on peut accepter quelques vrais négatifs (du spam classé comme légitime), mais on ne veut pas de de faux positif (des messages légitimes classifiés comme spam). Le taux de faux positifs étant de 0.046 et le taux de vrais négatifs étant de 0.109, ces critères sont relativement bien respectés, mais des progrès sont à faire. Plus de 90\% du spam est filtré, mais ils reste tout de même 5\% de messages légitimes qui partent au spam, ce qui oblige à vérifier régulièrement sa boite spam pour voir si un message important ne s'y trouve pas.\\
			
			La matrice de confusion de notre modèle se trouve à la figure \ref{tab:confMat22}.
			Les lignes représentent les deux classes (spam et légitime) correspondant à la réalité, et les colonnes les deux mêmes classes, mais pour le résultat de la classification. On peut donc retrouver tous les résultats précédents (mis à part l'air sous la courbe) à partir de cette matrice. En normalisant la matrice, plus celle-ci se rapproche d'une matrice diagonale, plus notre classificateur est performant (on n'a alors ni faux positif ni vrai négatif).
			
			}
			
			
			\begin{table}
			\centering
			\begin{tabular}{|c|c||c|}
			
			\hline 
			a & b & classification \\ 
			\hline 
			550 & 67 & a=1 \\ 
			\hline 
			44 & 903 & b=0 \\ 
			\hline 
			\end{tabular} 
			\caption{Matrice de confusion pour random forest}
			\label{tab:confMat22}			
			\end{table}			
			
		\end{homeworkSection}
			
		\begin{homeworkSection}{2.3}
			
			Donnez un exemple de contre-mesure de type Tokenization attack qu’un spammeur
			pourrait facilement utiliser afin de contourner un filtre basé uniquement sur la fréquence
			d’apparition de certains mots. Votre méthode ne doit pas modifier la signification du
			message et ne doit pas ajouter de nouveaux mots. Appliquez votre méthode au message
			suivant:\\

			\textit{DEAR RECEIVER,\\
			You have just received a Taliban virus. Since we are not so technologically
			advanced in Afghanistan, this is a MANUAL virus. Please click on
			this link (http://clickme.com) to delete all the files on your hard disk
			yourself and send this mail to everyone you know.
			Thank you very much for helping us.\\	\\
			-Taliban hacker.}\\

			\problemAnswer{
			
			D'après la documentation, "\textit{A word [...] is any string of alphanumeric characters bounded by non-alphanumeric characters or end-of-string}". Ainsi, il suffit de couper les mots ayant un odd-ratio élevé avec un caractère non alphanumérique afin qu'ils ne soient plus pris en compte lors de la détection. On peut par exemple le faire avec un tiret et un retour à la ligne. On considère que les mots de ce mail avec un odd-ratio élevé sont \textit{virus}, \textit{Afghanistan}, \textit{delete}, \textit{everyone}, \textit{helping} et \textit{hacker}.\\
			
			\textit{DEAR RECEIVER,\\
			You have just received a Taliban vi-\\rus. Since we are not so technologically
			advanced in Afgha-\\nistan, this is a MANUAL vi-\\rus. Please click on
			this link (http://clickme.com) to de-\\lete all the files on your hard disk
			yourself and send this mail to every-\\one you know.
			Thank you very much for hel-\\ping us.\\	\\
			-Taliban ha-\\cker.}\\
			
			
			}
					
		\end{homeworkSection}
		
			
	\end{homeworkProblem}
		\begin{homeworkProblem}[Apprentissage automatique]
	
	
		Une autre catégorie de méthodes qui a fait ses preuves en détection de spam consiste à
		s’inspirer de l’intelligence artificielle et de recourir à des techniques dites d’apprentissage
		automatique (machine learning). Ces méthodes ont l’avantage de s’adapter et
		d’apprendre pour continuellement améliorer leur performance. 

	
		\begin{homeworkSection}{3.1}

			Les méthodes d’apprentissage automatique peuvent être divisées en plusieurs
			catégories. Parmi les plus courantes, nous retrouvons l’apprentissage non-supervisé,
			semi-supervisé, et supervisé. Expliquez les caractéristiques de chacune de ces méthodes.

			\problemAnswer{
			Pour l'analyse et la catégorisation de grande quantité de données, les réseaux de neurones sont de plus en plus utilisés. 
			Pour pouvoir utiliser un réseau de neurones il faut cependant l'entrainer.
			Dans cette partie, nous allons présenter 3 méthodes pour entrainer un réseau de neurones.
			
			
			\paragraph{L'apprentissage non-supervisé.}
			Dans cette méthode, c'est l'algorithme qui effectue le tri de classe.
			Pour cela, il les traites comme un ensemble de variable aléatoire.
			L'apprentissage non-supervisé a donc l'avantage de ne pas nécessiter un expert.
			\cite{apprentissage_np}

			
			\paragraph{L'apprentissage supervisé.}
			L'apprentissage supervisé consiste à donné un jeu de donnée étiqueté au réseau de neurones.
			Contrairement à l'apprentissage non-supervisé, cette méthode nécessite d'étiqueter tout le jeu de donnée par un expert.
			Une fois le réseau "entrainé", il devrait être capable de catégorisé une entrée automatiquement.
			
			\paragraph{L'apprentissage semi-supervisé.}
			Cette méthode regroupe l'apprentissage supervisé et non-supervisé.			
			Dans l'apprentissage semi-supervisé, on utilise deux set de données, un étiqueté et un non étiqueté.
			Celui-ci permet de trier plus facilement des grands ensembles de données.
			\cite{apprentissage_sp}
			
			}
			
		\end{homeworkSection}
		
		\begin{homeworkSection}{3.2}

			La classification naïve bayésienne (naive bayes classifier) est un exemple de
			méthode qui peut être utilisée afin de résoudre des problèmes de classification par
			apprentissage supervisé. Appliquez cette méthode au fichier spambase afin de filtrer les
			messages en fonction des 57 variables continues. Utilisez 66\% des données pour la phase
			d’apprentissage.
			
			Évaluez et discutez des performances de votre modèle en termes de : taux de vrai positif,
			taux de faux positif, précision, sensibilité (recall), « F-measure » et l’aire sous la courbe
			(ROC area). Donnez la matrice de confusion.
	
			\problemAnswer{
			
			}
			
		\end{homeworkSection}
			
		\begin{homeworkSection}{3.3}

			Une autre méthode de classification, les forêts d’arbres décisionnels (random forest), consiste à
			effectuer un apprentissage sur plusieurs arbres de décisions. Appliquez
			cette méthode au fichier spambase et utilisez les 57 variables continues. Utilisez 66\% des
			données pour la phase d’apprentissage.
			Évaluez et discutez des performances de votre modèle en termes de : taux de vrai positif,
			taux de faux positif, précision, sensibilité (recall), « F-measure » et l’aire sous la courbe
			(ROC area). Donnez la matrice de confusion.

			\problemAnswer{
			
			}
					
		\end{homeworkSection}
		
		\begin{homeworkSection}{3.4}

			Donnez un exemple de contre-mesure de type Statistical attack qu’un spammeur
			pourrait utiliser afin d’échapper à un filtre basé sur la fréquence des mots en utilisant une
			méthode d’apprentissage automatique. Votre exemple de contre-mesure peut impliquer de
			modifier le contenu du message. Proposez une solution qui permettrait de contrecarrer
			cette contre-mesure.

			\problemAnswer{
			
			}
					
		\end{homeworkSection}
		
			
	\end{homeworkProblem}
		\begin{homeworkProblem}[Performance et optimisation]
	

		\begin{homeworkSection}{2.1}

			Comparez et discutez, en termes de performance, les résultats que vous avez obtenus
			pour les différentes méthodes utilisées (régression logistique, classification naïve
			bayésienne, forêts d’arbres décisionnels). Selon vos résultats, quelles méthodes semblent
			donner les meilleures performances pour le jeu de données spambase?

			\problemAnswer{
			
			}
			
		\end{homeworkSection}
		
		\begin{homeworkSection}{2.2}
			
			Nommez un avantage et un inconvénient d’utiliser des filtres basés sur
			l’apprentissage machine supervisé. Comparez avec l’apprentissage machine non
			supervisé en donnant un avantage et un inconvénient dans un contexte de détection de
			spam. Selon vous, est-ce qu’une méthode semble plus appropriée? Justifiez votre réponse
			et n’oubliez pas de donner vos références.
	
			\problemAnswer{
			
			}
			
		\end{homeworkSection}
			
		\begin{homeworkSection}{2.3}

			Le jeu de données spambase est principalement basé sur la fréquence d’apparition de
			certains mots et sur la présence de lettres majuscule. Donnez au moins deux autres
			exemples de caractéristiques (features) qui pourraient être prises en compte afin
			d’améliorer la performance des modèles de classification que vous avez développés.
			Donnez vos références. 

			\problemAnswer{
			
			}
					
		\end{homeworkSection}
		
			
	\end{homeworkProblem}

	\clearpage
	\bibliography{main}
	\bibliographystyle{ieee}

\end{document}
