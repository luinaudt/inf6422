\begin{abstract}

Bien que beaucoup aient mis en avant le risque posé par l'homogénéité au niveau logiciel, on ne sait pas précisément comment la diversité va réellement nuire à l'attaquant. Nous proposons d'apporter des éléments de réponse à l'impact de la diversité sur les attaques en fournissant deux modèles. L'un basé sur l'entropie qui nous permet de connaître le nombre moyen de machines infectées lors d'une attaque en fonction de la configuration logicielle du système, et prenant en compte le surcoût de cette diversité pour le défenseur. Dans le second, les différentes logiciels sont représentés et inter-connectés sous la forme d'un graphe, qui est ensuite parcouru par le \textit{malware.} Nous présentons enfin les résultats de la simulation de nos modèles sous différentes conditions, et montrons que la diversité logicielle permet bien de limiter les étendues des attaques.
\end{abstract} 