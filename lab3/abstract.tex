\begin{abstract}

Bien que beaucoup ont mis en avant le risque posé par l'homogénéité au niveau logiciel, on ne sait pas précisément comment la diversité va réellement nuire à l'attaquant. Nous essayons dans cet article d'apporter des éléments de réponses à l'impact de la diversité sur les attaques en fournissant deux modèles. L'un basé sur l'entropie qui nous permet de connaître le nombre moyen de machines infectées lors d'une attaque, en fonction de la configuration logicielle du système. Et notre second modèle va quand a lui s'intéresser à mettre en lumière le coût de la diversité auprès du défenseur, avec la proposition d'un modèle d'un système simple dans lesquels les différentes logiciels sont représentés et inter-connectés sous la forme d'un graphe. Et nous présentons  les résultats de la simulation de nos modèles sous différentes conditions.
\end{abstract} 