Le modèle présenté ici, est, à notre connaissance, le seul modèle permettant de mesurer le nombre d'infection en fonction de la
diversité de de l'ancienneté du système. Bien qu'un travail en amont soit nécessaire, les premiers résultats sont encourageants. 
\newline

En effet, nos expérimentations montrent, d'une part l'existence d'un lien entre l'entropie et le nombre de machines infectées et d'autre part que l'hypothèse d'une
relation linéaire entre ces quantités est raisonnable. \`A l'heure actuelle, le nombre d'applications ayant suffisamment de
déclinaisons pour que l'entropie d'un système puisse être supérieure à deux est en effet très faible. De ce fait, nos résultats
peuvent s'appliquer à de nombreux cas dans la vie réelle. 
\newline

Nous avons de plus présenté un modèle de coût, qui, une fois les constantes déterminées et une fonction d'évaluation du coût de
maintenance correctement définie, permettra de guider un administrateur réseau dans ses choix pour la maintenance de la
plateforme à sa charge.
\newline

Nous allons clôturer cet article, par le principe suivant: la diversité favorise la sécurité.
