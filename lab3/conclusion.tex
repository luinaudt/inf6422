Nous avons pu voir deux modèles présentant l'impact de la diversité logicielle. 

Le premier modèle présenté ici, est un modèle permettant de mesurer l'évolution du nombre moyen d'infections en fonction de la
diversité de de l'ancienneté des logiciels du système, et du coût lié à la diversité pour le défenseur. Le second modèle s'est quant à lui intéressé à système représenté sous la forme d'un graphe et inter-connecté pour étudier la propagation d'un malware en son sein. Selon les hypothèses que nous avons émis, nous avons pu voir que la diversité logicielle permet de limiter les étendues des attaques.

En effet, nos expérimentations montrent, d'une part l'existence d'un lien entre l'entropie et le nombre de machines infectées et d'autre part que l'hypothèse d'une
relation linéaire entre ces quantités est raisonnable. \`A l'heure actuelle, le nombre d'applications ayant suffisamment de
déclinaisons pour que l'entropie d'un système puisse être supérieure à deux est en effet très faible. De ce fait, nos résultats
peuvent s'appliquer à de nombreux cas dans la vie réelle. 

Nous avons de plus présenté un modèle de coût, qui, une fois les constantes déterminées et une fonction d'évaluation du coût de
maintenance définie selon ses besoins, permettra de guider un administrateur réseau dans ses choix pour la maintenance de la
plateforme à sa charge.

Finalement, nous avons montré que la diversité permet de réduire le nombre total de machines infectées.
Cette méthode est donc efficace notamment pour des système demandant une forte disponibilité ou pour se protéger des \textit{malware} de type \textit{ransomware}.

Étant donné qu'actuellement les navigateurs internet sont un des vecteurs principaux pour la transmission de \textit{malware}, il serait intéressant d'appliquer cela sur un cas réel pour ce type de logiciel et de comparer les résultats de l'infection avec notre modèle théorique.

Nous pouvons donc conclure que la diversité logicielle a une influence sur l'étendue d'une infection, et qu'elle a des effets négatifs pour l'attaquant. Et qu'ainsi la diversité favorise la sécurité. 