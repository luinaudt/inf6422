Le modèle que nous avons présenté est, à notre connaissance, le seul modèle permettant de mesurer le nombre d'infection en fonction de la
diversité de de l'ancienneté d'un système. %Bien qu'un travail en amont soit nécessaire, les premiers résultats sont encourageants. 
Ce modèle demande encore à être validé dans des cas concrets.
\newline
Nos expérimentations ont montré, l'existence d'un lien entre l'entropie et le nombre de machines infectées.
Nous avons également déterminé que cette relation peut être approximée par une fonction linéaire. 
\`A l'heure actuelle, le nombre d'applications ayant suffisamment de déclinaisons pour que l'entropie d'un système puisse être supérieure à deux est très faible. 
De ce fait, l'approximation linéaire reste bonne avec des cas réels. 
\newline
De plus, nous avons introduit la notion de coût à notre modèle.
Il est donc possible, une fois les constantes définies et la fonction d'évaluation du coût de la diversité déterminée, de guider l'administrateur réseaux dans la priorisation des choix de maintenance pour les plateformes à sa charge.
\newline
Finalement, nous avons montré que la diversité permet de réduire le nombre total de machines infectées.
Cette méthode est donc efficace notamment pour des système demandant une forte disponibilité ou pour se protéger des \textit{malware} de type \textit{ransomware}.
\\
Étant donné qu'actuellement les navigateurs internet sont un des vecteurs principaux pour la transmission de \textit{malware}, il serait intéressant d'appliquer un cas réel pour ce type de logiciel et de comparer les résultats de l'infection avec notre modèle théorique.