Nous avons proposé dans cet article un modèle sur le taux d'infection en fonction de la diversité logicielle. Ce modèle bien
qu'intéressant comporte quelques limitations sur lesquelles une recherche ultérieure serait souhaitable. En premier lieu nous avons
considéré qu'un attaquant serait plus apte à maximiser le nombre d'infections plutôt que la probabilité d'infecter une machine.
Cette hypothèse est raisonnable dans de nombreuses situations mais elle peut sembler inappropriée dans certains cas. En effet, si
l'attaquant
cherche à récupérer des informations confidentielles mais a priori présentes sur un petit nombre d'ordinateurs au sein d'une
compagnie, il sera alors plus intéressant pour lui de maximiser la probabilité d'infecter ces machines et non le nombre total de machines infectées.
\newline

Un autre point important pour compléter nos résultats est la détermination de constantes. Nous avons présenté un
modèle de coût, qui étudier de manière plus approfondie pourrait avoir de nombreuses applications. En particulier, il permettrait de chiffrer un
coût lors d'un audit et de donner des métriques tangibles qui pourraient servir de base pour déterminer un plan
d'action. Une telle recherche nécessiterait des compétences économiques pour évaluer les différents coûts mais aussi un
savoir-faire technique pour déterminer la dangerosité d'un système.
\newline

Nos équations se basent sur une constante $\lambda$ dont la valeur doit être déterminer afin d'obtenir des résultats numériques. Une
études de champs est souhaitable pour approximer le $\lambda$ moyen d'une attaque. Il serait également intéressant de déterminer un
$\lambda$ pour différents types d'attaques auxquels nous pouvons être confrontés.
\newline

Nos résultats se basent sur l'hypothèse qu'un malware ne peut exploiter qu'une vulnérabilité et que cette vulnérabilité
ne peut affecter deux logiciels distincts. Nous aurions pu prendre en compte plutôt une vulnérabilité qu’une version de logiciel comme cela a été fait par  Neti et al.\cite{softwareDiversity:Security}
 Il serait intéressant de voir l'impact de ce hypothèse notre modèle.

Enfin, nous pourrions d'un travail futur de confronter nos résultats à des situations concrètes via une étude de mesures issues de
données réelles. Il est important de vérifier les résultats des simulations en les mettant en corrélation à des données tirées de systèmes réels. 
\newline

Bien qu'ayant ses limites, nos résultats restent valables dans les hypothèses annoncées. Ces dernières correspondent
qu'à certaines situations possibles et sont valables dans ces cas-là. Et, les simulations confirment la relation
linéaire entre l'entropie et le nombre d'infections. 
