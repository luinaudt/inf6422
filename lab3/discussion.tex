Nous avons proposé dans cet article un modèle sur le taux d'infection en fonction de la diversité logicielle. Ce modèle bien
qu'intéressant comporte malheureusement quelques points où une recherche ultérieure serait souhaitable. En premier lieu nous avons
considéré qu'un attaquant serait plus apte à maximiser le nombre d'infections plutôt que la probabilité d'infecter une machine.
Cette hypothèse est raisonnable dans de nombreuses situations mais elle peu sembler inappropriée dans certains cas. En effet, si
cherche à récupérer des informations confidentielles mais a priori présentes sur un grand nombre d'ordinateurs au sein d'une
compagnie il sera alors plus intéressant pour lui de maximiser la probabilité et nom le nombre de victimes.
\newline

Un autre point important pour une future amélioration de nos résultats est la détermination de constantes. Nous avons présenté un
modèle de coût, qui s'il venait à être abouti aurait de nombreuses applications. En particulier, il permettrait de chiffrer un
coût lors d'un audit et donc de donner des métriques tangibles, lesquelles pourraient servir de base pour déterminer un plan
d'action. Malheureusement, une telle recherche nécessiterait des compétences économiques pour évaluer un coût mais aussi un
savoir-faire technique pour jauger la dangerosité d'un système.
\newline

De plus, nous basons nos équations sur une constante $\lambda$ qui faudrait déterminer afin d'obtenir des résultats numériques. Une
études de champs serait don souhaitable pour approximer le $\lambda$ moyen d'une attaque. On pourrait également déterminer un
$\lambda$ pour différents types d'attaques auxquels on pense être confronté.
\newline

Par ailleurs, nos résultats se basent sur l'hypothèse qu'un malware ne peut exploiter qu'une vulnérabilité et qu'une vulnérabilité
ne peut affecter deux logiciels distincts. Ces deux hypothèses sont bien sûr réductrices et fausses. Un travail supplémentaire est
donc nécessaire pour déterminer si ces hypothèses sont source d'un biais trop grand. Si c'est le cas, il faudrait également
travailler sur un modèle qui prendrait les prendrait en compte.

Enfin, d'une manière générale, il serait souhaitable de confronter nos résultats à la réalité via une étude de mesures issues de
données réelles. Il est nécessaire de vérifier les résultats des simulations en les comparant à des données réelles. 
