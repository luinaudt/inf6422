La diversité est un principe que l'on retrouve dans plusieurs domaines. Par exemple, la diversité des espèces est l’un des fondements de la nature. Au niveau de l’agriculture nous avons des exemples de monoculture, qui ont entrainé de graves famines une fois le plantations infectées par une maladie et détruites en très grande partie. Nous pouvons notamment voir l’exemple historique de la grande famine suite à l’apparition d’un parasite infectant les pommes de terres, le mildiou, qui a touché l’Europe à la moitié du 19ème siècle, et de manière beaucoup plus ravageuse encore l'Irlande, où l’agriculture reposait essentiellement sur la production de pommes de terre.

Dans le domaine informatique, la monoculture présente des risques notamment sur la rapidité et l’ampleur de l’infection \cite{risksOfMonoculture}. La dépendance à un seul genre de logiciel permet une plus grande facilité de propagation à travers les failles partagés sur les différentes parties d'un système.
Un fait plus récent a attiré notre attention, peu de temps après les attentats qui ont eu lieu à Paris en Janvier 2015, de nombreux sites de municipalités française ont été victime d’attaques. Plusieurs centaines de sites ont été piratés parmi les plus de 35000 communes qui compte la France \cite{communes_INSEE}. A la suite de ces attaques, un audit a été effectué auprès des municipalités \textit{La gazette des communes} . On y apprend qu’environ la moitié des communes disposent de leur propre site avec notamment une très grande biodiversité logicielle, tant au niveau des logiciels que des versions. Ce facteur a-t-il permis de réduire l’ampleur des attaques réussies ?

Nous pouvons voir aujourd’hui que la diversité n’est pas présente au sein des systèmes d’exploitations, où Windows domine le marché avec plus de 90\% des parts, mais aussi au niveau des navigateurs web où son navigateur propriétaire est également prédominant, avec une cependant une tendance à la baisse lors des dernières années \cite{netMarketShare}. 
Dans notre étude, nous nous intéressons à la diversité logicielle, ce que nous définirons par la suite comme l'utilisation d'un ensemble de différents logiciels, mais servant à la même tâche (e.g. navigateur web, serveur web, traitement de texte, ...). Nous définirons alors une entropie des systèmes par rapport aux différentes versions présentes. Celle-ci servira de base pour définir un modèle de propagation d'un \textit{malware} donné dans un système quelconque, en fonction de l'ancienneté moyenne des logiciels qui y sont installés et de leur entropie. Ce modèle prendra également en compte le coût de maintenance du parc informatique pour le défenseur en fonction de l'entropie logicielle de celui-ci, afin d'essayer de trouver le meilleur compromis sur la diversité logicielle.
Afin de valider notre modèle, nous ferons dans un second temps une simulation basée sur un graphe, chaque nœud représentant une machine. Le \textit{malware} essayera ensuite de parcourir le graphe en se propageant par l'intermédiaire d'un lien réseau entre deux machines étant équipées du même logiciel. Nous présenterons ensuite nos résultats, et en tirerons des conclusions sur les situations dans lesquelles il est avantageux pour le défenseur d'investir dans la diversité logicielle de son système afin de le protéger.
