La diversité est un terme vague mais présent dans la nature où l’on parle de diversité des espèces qui est l’un des fondements de la nature. Au niveau de l’agriculture nous avons des exemples de monoculture, ou ce qui y ressemblait fortement, qui ont entrainé de grave conséquence une fois mise en infecté par des agents pathogènes. Nous pouvons notamment voir l’exemple historique de la grande famine suite à l’apparition d’un parasite infectant les pommes de terres, le mildiou, qui a touché l’Europe à la moitié du 19ème siècle, et de manière beaucoup plus ravageuse en Irlande où l’agriculture reposait essentiellement sur la production de pomme de terre.

Dans le domaine informatique, la monoculture présente des risques notamment sur la rapidité et l’ampleur de l’infection \cite{risksOfMonoculture}. La dépendance à un seul genre de logiciel, permet une plus grande facilité de propagation à travers les failles partagés sur les différentes parties du système.
Un fait plus récent a attiré notre attention, peu de temps après les attentats qui ont eu lieu à Paris en Janvier 2015, de nombreux sites de municipalité française ont été victime d’attaques. Plusieurs centaines de sites ont été piratés parmi les plus de 35000 communes qui compte la France \cite{communes_INSEE}. A la suite de ces attaques, un audit a été effectué auprès des municipalités \textit{La gazette des communes} . On y apprend qu’environ la moitié des communes disposent de leur propre site avec notamment une très grande biodiversité logicielle, tant au niveau des logiciels que des versions. Ce facteur a-t-il permis de réduire l’ampleur des attaques réussites ?

Nous pouvons voir aujourd’hui que la diversité n’est pas présente au sein des systèmes d’exploitations, où Windows domine le marché avec plus de 90\% des parts, mais aussi au niveau des navigateurs web où son navigateur propriétaire dominent le marché, avec une cependant une tendance à la baisse lors des dernières années \\\\\cite{netMarketShare}. 
Dans notre étude, nous nous intéressons à la diversité des logiciels, où nous faisons référence à des versions de logiciels modélisant des logiciels ayant des finalités d’utilisations similaires. Nous nous intéressons donc l’entropie des systèmes par rapport aux différentes versions présentes. Nous introduisons un modèle prenant en compte l’ancienneté des versions pour essayer d’en tirer le nombre d’infection en fonction du temps. Notre second modèle va s’intéresser à mettre en avant le coût que la diversité logicielle aurait pour protéger un système. 
Et nous allons présenter les résultats de nos modélisations ainsi qu’une simulation en laboratoire qui visera à montrer l’impact de l’hétérogénéité sur un système.
