En janvier 2015, à la suite des attentats à Charlie Hebdo, une vague d'attaques a ciblé les sites des municipalités françaises\cite{courrier}.
Sur les 36658 communes que comptait la France en 2015 d'après l'INSEE\cite{communes_INSEE}, plusieurs centaines ont vu leur site web piraté.
Doit-on en déduire que les sites web des communes françaises sont mal protégés ? Si oui, quelle en est la cause ?
Ou au contraire, la vague d'attaques aurait-elle pu être encore plus vaste ? Si oui, quel élément l'a-t-elle freinée ?\\
Pour répondre à ces questions, le magazine \textit{La gazette des communes} a effectué en février 2015 un audit des sites webs municipaux français\cite{gazette}. On y découvre qu'environ 15000 communes disposent de leur propre site, mais surtout que ceux-ci présentent une très grande biodiversité logicielle, tant au niveau des logiciels utilisés que de leurs versions.
Nous allons donc par la suite évaluer l'impact de ce facteur environnemental sur les critères de performance d'un attaquant et d'un défenseur (les communes). Nous considérerons que le système de défense est inexistant. Le défenseur pourra seulement influer sur l'environnement moyennant un certain coût de migration. L'attaquant, lui, pourra adapter son modèle d'attaque à l'environnement dans une certaine mesure : celui-ci n'est pas toujours connu (les sites web ont la possibilité de masquer leur configuration), et cela a un certain coût (recherche ou achat de nouveaux \textit{exploits}).\\
Dans un premier temps, nous développerons un modèle mathématique. Nous introduirons une entropie logicielle caractérisant la "biodiversité" de notre système, puis nous introduirons des fonctions de changement d'état, qui impliquent une certaine énergie et donc un certain coût pour le défenseur. Nous introduirons également les équations du côté de l'attaquant : l'efficacité de son attaque dépendra, tout comme en thermodynamique pour l'efficacité d'une machine thermique, de l'entropie du système.
Dans un second temps, nous appliquerons notre modèle aux données obtenus à partir de l'article de \textit{La gazette}, qui on été publiées en Open Data sur le site gouvernemental \textit{data.gouv.fr}\cite{data.gouv}.