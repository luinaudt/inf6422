En janvier 2015, à la suite des attentats à Charlie Hebdo, une vague d'attaques a ciblé les sites des municipalités françaises\cite{courrier}.
Sur les 36658 communes que comptait la France en 2015 d'après l'INSEE\cite{communes_INSEE}, plusieurs centaines ont vu leur site web piraté.
Doit-on en déduire que les sites web des communes françaises sont mal protégés ? Si oui, quelle en est la cause ?
Ou au contraire, la vague d'attaques aurait-elle pu être encore plus vaste ? Si oui, quel élément l'a-t-elle freinée ?\\
Pour répondre à ces questions, le magazine \textit{La gazette des communes} a effectué en février 2015 un audit des sites webs municipaux français\cite{gazette}. On y découvre qu'environ 15000 communes disposent de leur propre site, mais surtout que ceux-ci présentent une très grande biodiversité logicielle, tant au niveau des logiciels utilisés que de leurs versions.
Nous allons donc par la suite évaluer l'impact de ce facteur environnemental sur les critères de performance d'un attaquant et d'un défenseur (les communes). Nous considérerons que le système de défense est inexistant. Le défenseur pourra seulement influer sur l'environnement moyennant un certain coût de migration. L'attaquant, lui, pourra adapter son modèle d'attaque à l'environnement dans une certaine mesure : celui-ci n'est pas toujours connu (les sites web ont la possibilité de masquer leur configuration), et cela a un certain coût (recherche ou achat de nouveaux \textit{exploits}).\\
Dans un premier temps, nous présenterons une analogie avec la thermodynamique. Nous introduirons une entropie logicielle caractérisant l’hétérogénéité de notre système, puis nous introduirons des fonctions de changement d'état, qui impliquent une certaine énergie et donc un certain coût pour le défenseur. Nous introduirons également les équations du côté de l'attaquant : l'efficacité de son attaque dépendra, tout comme en thermodynamique pour l'efficacité d'une machine thermique, de l'entropie du système.\\
Dans un second temps, nous développerons deux modèles mathématiques. Le premier se basera sur des données expérimentales, obtenues à partir de l'article de \textit{La gazette} et qui on été publiées en Open Data sur le site gouvernemental \textit{data.gouv.fr}\cite{data.gouv}. Nous définirons un coût pour l'attaquant basé sur les vulnérabilités présentes sur les différentes versions du système. Il sera basé sur le nombre de versions, et le nombre et l'ancienneté de chaque vulnérabilité connue. Il y aura également un coût pour le défenseur, basé sur le nombre d'attaques et le coût de mise à jour de son système. Inversement, nous pouvons également définir un gain pour l' attaquant et le défenseur selon chacune de leur stratégie respective (respectivement nombre de vulnérabilités exploitées et nombre de mises à jour). Après calculs et application de la théorie des jeux, nous obtiendrons un équilibre de Nash, qui nous indiquera le nombre optimal de mises à jour à effectuer sur notre système.\\
Le second modèle mathématique sera plus théorique. Nous définirons la loi de risque d'attaque d'une version donnée d'un logiciel en fonction de son ancienneté, et la probabilité d'attaque correspondante du système complet au temps t. Nous poserons également le coût de mise à jour incrémentale du système au cours du temps. Nous en déduirons un système d'équations différentielles. En fixant la probabilité d'attaque souhaitée par le défenseur, nous pourrons alors en déduire l'effort à investir dans les mises à jour au cours du temps afin de garder cette probabilité fixe, et le coût correspondant.