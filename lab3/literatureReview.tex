L'étude des vulnérabilités au sein des logiciels n'est pas chose nouvelle. Des études quantitatives cherchant à modéliser le
nombre de failles de sécurités découvertes au cours du temps ont déjà été menées \cite{vulnerabilityDiscovery}. Et ce, avec des
résultats plus que satisfaisants. Cet article s'intéresse à la possibilité de modéliser le taux de découvertes de vulnérabilités
afin de pouvoir anticiper et prévoir à l'avance des attaques et des intrusions afin de consacrer les ressources optimales pour la résolution des failles qui
seraient découvertes. Il y a donc un intérêt pour une gestion améliorer des effectifs. Une autre étude, plus centrée sur l'anticipation du
nombre de failles dans un programme, a vu le jour \cite{assessingVulnerabilities}. Cet article introduit des modèles
permettant de caractériser le nombre de vulnérabilités que l'on trouvera si l'on consacre un certain "effort" à leur recherche.

La diversité logicielle a été au cœur de nombreuses recherches, soulignant notamment son utilité en matière de réduction du risque \cite{softwareDiversityStateOfTheArt} et de prévention des attaques critiques. L'analyse de la stratégie de la diversité logicielle pour la sécurité de l’information a permis de limiter certains risques. Des études ont conduit à la formulation d'un modèle permettant d'inférer le montant optimal de l’investissement pour une compagnie en prenant en compte l’uniformité et de l’interopérabilité des différents systèmes\cite{informationSecurity}. 
Cette diversité a été étudié parmi plusieurs types de logiciels, comme les antivirus \cite{DiversityForSecurityAntivirus}montrant l’efficacité couplée de ceux-ci pour détecter les malwares. Des tests ont été menés parmi un nombre défini d'antivirus du marché, et on permit de mettre en évidence l'incapacité de ceux-ci à détecter individuellement toute une panoplie de malware.
 Au niveau du réseau, la diversité permet aussi de se prémunir par rapport aux malwares, nous pouvons ajouter une diversité dans notre réseau pour réduire la vitesse de propagation d’une potentielle infection \cite{OptimisingNetwork} contre par exemple des vers. La diversité réseau a été longtemps étudié comme un mécanisme de sécurité contre des attaques variées. La modélisation de la diversité logicielle a été proposé comme une mesure de sécurité en designant et en évaluant une série de métriques, deux métriques complémentaires ont notamment été proposé pour évaluer la fiabilité des réseaux contre les failles critiques \cite{networkDiversity}.

L'existence de malwares utilisant plusieurs vulnérabilités pour se propager est déjà connu. On se souviendra ainsi du
vers \textit{Nimda}, qui utilisait 5 vulnérabilités différentes. Du point de vue de la défense, l'idée d'utiliser différents
logiciels, afin de varier les vulnérabilités et donc de limiter les possibilités de l'attaquant n'est pas nouvelle, et plusieurs
travaux ont tenté de caractériser le gain induit par une telle approche. En particulier, \cite{softwareDiversity:Security}
applique la notion d'entropie à des graphes bipartis afin de mesurer la diversité d'un système. Nous avons pu voir apparaître des systèmes de mesures de cette diversité logicielle \cite{softwareDiversityMetrics} pour nous permettre de mieux nous rendre compte de son apport en terme de sécurité.

L'exploitation de la diversité des logiciels permet ainsi d'améliorer la fiabilité des systèmes en réduisant le nombre de faille communes sur les différents logiciels, qui ne sont pas forcément présent dans des systèmes d'exploitations différents ou de versions différentes.\cite{softwareDiversityPracticalStatistics}
Elle permet d'augmenter la sécurité des communications grâce à une demande plus importante au niveau de l'effort de l'attaquant pour corrompre les différents éléments, comme illustré en utilisant un taux de maximal de diversité parmi les différents éléments du réseau et de logiciels\cite{maximalRatio}. 

Cette diversité implique des conséquences à plusieurs niveaux et doit aussi être prise en compte lors du développement logiciel\cite{processDiversity}. Il est nécessaire de prendre en considération les parties essentiels et déterminer les dépendances d'un logiciel pour avoir la possibilité d'améliorer de façon fiable dans le temps les logiciels qui pourront avoir des bases plus fiables. Il faut repenser alors en conséquent la manière dont vont communiquer les différents éléments du réseau, ce qui impact de manière non négligeable la réponse aux besoins de l'utilisateur, le design et l'implémentation du logiciel.
La diversité logiciel permet donc une meilleure prévention des attaques, notamment lorsqu'elle est utilisée dans la tolérance à l'intrusion\cite{osDiversity}. L'utilisation de systèmes diversifié laissant certaines intrusions permet d’être sûr du comportement en cas d’attaques et d’intrusions. Les gains de sécurité dépendent directement des composants et de leur implémentation. L'usage de différents systèmes d'opération peut être une technique utile pour réduire le risque de failles critiques. 
Nous pouvons même pousser l'usage de la diversité au niveau des données, Les applications critiques nécessitent des logiciels extrêmement fiables.  La diversité dans les espaces de stockage autorise une certaine tolérance aux pannes logiciels\cite{dataDiversity}, notamment par l'utilisation de diverses sources de stockage des données.


