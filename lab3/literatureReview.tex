De nombreuses études se sont intéressées à la diversité au niveau informatique, ainsi pour avoir une meilleure approche et mieux comprendre dans quelles circonstances la diversité peut nuire à l’attaquant, et peut être une bonne stratégie de défense, nous avons besoin de bien définir le concept de diversité. 
[suite à venir]

Kim et al se sont intéressés à l’étude quantitative du nombre de failles découvert au cours du temps \cite{vulnerabilityDiscovery}. Leur étude a porté sur deux logiciels open-source répandu : Apache HTTP Web Server et Mysql DBMS, pour identifier la relation entre la diversité des versions d’un logiciel et la découverte de vulnérabilité.

La diversité peut être représenté au niveau réseau, pour mieux se prémunir par rapport aux malwares. La diversité réseau permet alors de réduire la vitesse de propagation lors d’une potentielle infection \cite{OptimisingNetwork}. 

La diversité réseau a été longtemps étudiée comme un mécanisme de sécurité contre des attaques variées. La modélisation de la diversité logicielle a été proposée comme une mesure de sécurité en désignant et en évaluant une série de métriques. Deux métriques complémentaires ont notamment été proposées pour évaluer la fiabilité des réseaux contre les failles critiques \cite{networkDiversity}.

Du point de vue de la défense, l'idée d'utiliser différents logiciels, afin de limiter les vulnérabilités communes des différentes machines d’un système et donc de limiter les conséquences des attaques a été déjà étudié, plusieurs
travaux ont tenté de caractériser le gain induit par une telle approche. En particulier, \cite{softwareDiversity:Security}

L'exploitation de la diversité des logiciels a été présenté comme un moyen d'améliorer la fiabilité des systèmes en réduisant le nombre de failles communes sur les différents logiciels, et ainsi réduire l’étendu des infections grâce notamment à l’utilisation de systèmes d'exploitations différents ou des versions différentes.\cite{softwareDiversityPracticalStatistics}
Des malwares exploitant plusieurs vulnérabilités ont déjà été rencontré, notamment \textit{Nimda}, qui utilisait 5 vulnérabilités différentes. 
Elle permet aussi d'augmenter la sécurité des communications grâce à une complexité accrue pour un attaquant qui aura besoin d’un effort de plus important pour corrompre les différents éléments, comme illustré en utilisant un taux maximal de diversité parmi les différents éléments du réseau et de logiciels\cite{maximalRatio}. 

Les attaques ont des conséquences financières pour les compagnies, des études se sont intéressés à formuler des modèles permettant d’inférer le montant optimal de l’investissement pour une compagnie en prenant en compte l’uniformité et l’interopérabilité des différents systèmes \cite{informationSecurity}. Et ainsi, consacrer les ressources optimales pour la résolution des failles qui seraient découvertes. 

Il y a aussi possibilité de prendre en compte une diversité dans le processus même de développement logiciel \cite{processDiversity}. 
Lors de la conception d’un logiciel, il est important de prendre déterminer les parties fondamentales du logiciel et déterminer les dépendances, pour une évolution simplifier du logiciel dans le temps. Il faut repenser alors en conséquent la manière dont vont communiquer les différents éléments du réseau, ce qui impact de manière non négligeable la réponse aux besoins de l'utilisateur, le design et l'implémentation du logiciel.

La diversité dans les espaces de stockage  autorise une certaine tolérance aux pannes logiciels\cite{dataDiversity}, notamment par l'utilisation de diverses sources de stockage des données.

