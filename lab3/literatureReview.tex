L'études des vulnérabilités au sein des logicielles n'est pas chose nouvelle. Des études quantitatives cherchant à modéliser le
nombre de failles de sécurités découvertes au cours du temps ont déjà été menées \cite{vulnerabilityDiscovery}. Et ce, avec des
résultats plus que satisfaisants. Il s'agissait dans cet article de parvenir à modéliser le taux de découvertes de vulnérabilités
afin de pouvoir l'anticiper et prévoir à l'avance de consacrer un certain nombre de personnes à la résolution des failles qui
serraient découvertes. Il s'agissait donc plutôt de mieux gérer ses effectifs. Une autre étude, plus centrée sur l'anticipation du
nombre de failles dans un programme, a vu le jour \cite{assessingVulnerabilities}. Cet article introduit également des modèles
permettant de caractériser le nombre de vulnérabilités que l'on trouvera si l'on consacre un certain "effort" à leur recherche.
Par ailleurs, l'existence de malware utilisant plusieurs vulnérabilités pour se propager est déjà connu. On se souviendra ainsi du
vers \textit{Nimda}, qui utilisait 5 vulnérabilités différentes. Du point de vue de la défense, l'idée d'utiliser différents
logicielles, afin de varier les vulnérabilités et donc de limiter les possibilités de l'attaquant n'est pas nouvelle, et plusieurs
travaux ont tenté de caractériser le gain induit par une telle approche. En particulier, \cite{softwareDiversity:Security}
applique la notion d'entropie à des graphes bipartis afin de mesurer la diversité d'un système.
