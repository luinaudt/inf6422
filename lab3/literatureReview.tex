L'études des vulnérabilités au sein des logicielles n'est pas chose nouvelle. Des études quantitatives cherchant à modéliser le
nombre de failles de sécurités découvertes au cours du temps ont déjà été menées \cite{vulnerabilityDiscovery}. Et ce, avec des
résultats plus que satisfaisants. Il s'agissait dans cet article de parvenir à modéliser le taux de découvertes de vulnérabilités
afin de pouvoir l'anticiper et prévoir à l'avance de consacrer un certain nombre de personnes à la résolution des failles qui
serraient découvertes. Il s'agissait donc plutôt de mieux gérer ses effectifs. Une autre étude, plus centrée sur l'anticipation du
nombre de failles dans un programme, a vu le jour \cite{assessingVulnerabilities}. Cet article introduit également des modèles
permettant de caractériser le nombre de vulnérabilités que l'on trouvera si l'on consacre un certain "effort" à leur recherche.
