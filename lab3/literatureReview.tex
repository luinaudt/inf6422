De nombreuses études se sont intéressées à la diversité au niveau informatique afin d'en avoir une meilleure approche et de mieux comprendre dans quelles circonstances celle-ci peut nuire à l’attaquant, et peut alors être une bonne stratégie de défense.

Kim et al se sont intéressés à l’étude quantitative du nombre de failles découvertes au cours du temps \cite{vulnerabilityDiscovery}. Leur étude a porté sur deux logiciels open-source répandus : Apache HTTP Web Server et Mysql DBMS, pour identifier la relation entre la diversité des versions d’un logiciel et la découverte de vulnérabilités.

La diversité peut être représentée au niveau réseau, pour mieux se prémunir des malwares. Elle permet alors de réduire la vitesse de propagation lors d’une potentielle infection \cite{OptimisingNetwork}.La diversité réseau a été longtemps étudiée comme un mécanisme de sécurité contre des attaques variées.

La modélisation de la diversité logicielle a été proposée comme une mesure de sécurité en désignant et en évaluant une série de métriques. Deux métriques complémentaires ont notamment été proposées pour évaluer la fiabilité des réseaux contre les failles critiques \cite{networkDiversity}.

Du point de vue de la défense, l'idée d'utiliser différents logiciels, afin de limiter les vulnérabilités communes des différentes machines d’un système et donc de limiter les conséquences des attaques a déjà été mise en avant, plusieurs travaux ayant tenté de caractériser le gain induit par une telle approche\cite{softwareDiversity:Security}.

L'exploitation de la diversité des logiciels a été présentée comme un moyen d'améliorer la fiabilité des systèmes en réduisant le
nombre de failles communes sur les différents logiciels, et ainsi réduire l’étendue des infections grâce notamment à l’utilisation de systèmes d'exploitations différents ou de versions différentes,\cite{softwareDiversityPracticalStatistics}
contre notamment des malwares qui peuvent exploiter plusieurs vulnérabilités, comme \textit{Nimda}, qui utilisait cinq vulnérabilités différentes. 
Elle permet également d'augmenter la sécurité des communications grâce à une complexité accrue pour un attaquant qui aura besoin d’un effort plus important pour corrompre les différents éléments, en utilisant par exemple un taux maximal de diversité\cite{maximalRatio} parmi les différents éléments du réseau et de logiciels. 

Les attaques ont des conséquences financières pour les compagnies, des études se sont intéressées à formuler des modèles permettant d’inférer le montant optimal de l’investissement pour une compagnie en prenant en compte l’uniformité et l’interopérabilité des différents systèmes \cite{informationSecurity}. Et ainsi, consacrer les ressources optimales pour la résolution des failles qui seraient découvertes. 

Il y a aussi possibilité de prendre en compte une diversité dans le processus même de développement logiciel \cite{processDiversity}. 
Lors de la conception d’un logiciel, il est important de déterminer les parties fondamentales du logiciel et les dépendances, pour une évolution simplifiée du logiciel dans le temps. Il faut repenser alors en conséquence la manière dont vont communiquer les différents éléments du réseau, ce qui impacte de manière non négligeable la réponse aux besoins de l'utilisateur, ainsi que la conception et l'implémentation du logiciel.

La forme la plus courante de diversité au niveau de la conception des logiciels est le N-version programming\cite{ NversionProgramming }, dans lequel les développeurs vont écrire des composantes de façon répétitive et de manière conventionnelle. L’hypothèse derrière la diversité au niveau de la conception est que la séparation des équipes de développeurs va pouvoir révéler différents bugs dans les différentes versions. Même si le logiciel va alors rencontrer des erreurs, l’idée est que les différentes versions font entraîner des erreurs à des moments différents et de manière différente. 

La diversité dans les espaces de stockage autorise une certaine tolérance aux pannes logiciel\cite{dataDiversity}, notamment par l'utilisation de diverses sources de stockage des données. La diversité est aussi présente au niveau du design des composantes pour éviter les défauts de fonctionnement\cite{SecurityThroughDiversity}.

Comme nous pouvons le voir, la diversité en informatique peut se traduire sous différentes formes, dans notre étude nous nous sommes intéressés à la diversité au niveau logiciel pour étudier l'impact que cela peut avoir contre les attaques. Dans la partie suivante, nous allons présenter notre méthodologie et les hypothèses que nous avons pris en compte. 
