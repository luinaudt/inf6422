\documentclass[conference]{IEEEtran}
%\documentclass[conference, final]{IEEEtran}

\ifCLASSINFOpdf
  % \usepackage[pdftex]{graphicx}
  % declare the path(s) where your graphic files are
  % \graphicspath{{../pdf/}{../jpeg/}}
  % and their extensions so you won't have to specify these with
  % every instance of \includegraphics
  % \DeclareGraphicsExtensions{.pdf,.jpeg,.png}
\else
  % or other class option (dvipsone, dvipdf, if not using dvips). graphicx
  % will default to the driver specified in the system graphics.cfg if no
  % driver is specified.
  % \usepackage[dvips]{graphicx}
  % declare the path(s) where your graphic files are
  % \graphicspath{{../eps/}}
  % and their extensions so you won't have to specify these with
  % every instance of \includegraphics
  % \DeclareGraphicsExtensions{.eps}
\fi

\usepackage[utf8]{inputenc}
\usepackage{cite}
\usepackage{amsmath}
\usepackage[acronym]{glossaries}
\usepackage{lipsum, adjustbox}

\usepackage{pgf}
\usepackage{tikz}
\usetikzlibrary{arrows}



\usepackage{algorithmic}
\usepackage{array}

\ifCLASSOPTIONcompsoc
  \usepackage[caption=false,font=normalsize,labelfont=sf,textfont=sf]{subfig}
\else
  \usepackage[caption=false,font=footnotesize]{subfig}
\fi

%\usepackage{fixltx2e}
%\usepackage{stfloats}
%\usepackage{dblfloatfix}
\usepackage{url}

\hyphenation{op-tical net-works semi-conduc-tor}


\usepackage[]{todonotes}
%\addbibresource{main.bib}

\begin{document}

\title{Étude de l'influence de la biodiversité logicielle d'un système sur sa probabilité d'infection}

\author{
\IEEEauthorblockN{Philippe Troclet}
\IEEEauthorblockA{
%Email: 
philippe.troclet@polymtl.ca}
\and

\IEEEauthorblockN{Alexandre Mao}
\IEEEauthorblockA{
%Email: 
@}
\and
\IEEEauthorblockN{Paul Berthier}
\IEEEauthorblockA{
%Email: 
paul.berthier@polymtl.ca}
\and
\IEEEauthorblockN{Thomas Luinaud}
\IEEEauthorblockA{
%Email: 
thomas.luinaud@polymtl.ca}
}

\maketitle

\begin{abstract}

Bien que beaucoup ont mis en avant le risque posé par l'homogénéité au niveau logiciel, on ne sait pas précisément comment la diversité va réellement nuire à l'attaquant. Nous essayons dans cet article d'apporter des éléments de réponses à l'impact de la diversité sur les attaques en fournissant deux modèles. L'un basé sur l'entropie qui nous permet de connaître le nombre moyen de machines infectées lors d'une attaque, en fonction de la configuration logicielle du système. Et notre second modèle va quand a lui s'intéresser à mettre en lumière le coût de la diversité auprès du défenseur, avec la proposition d'un modèle d'un système simple dans lesquels les différentes logiciels sont représentés et inter-connectés sous la forme d'un graphe. Et nous présentons  les résultats de la simulation de nos modèles sous différentes conditions.
\end{abstract} 
%\cite{ArchForCounter}

\IEEEpeerreviewmaketitle

\section{Introduction}
La diversité est un principe que l'on retrouve dans plusieurs domaines. Par exemple, la diversité des espèces est l’un des fondements de la nature. Au niveau de l’agriculture nous avons des exemples de monoculture, qui ont entrainé de graves famines une fois le plantations infectées par une maladie et détruites en très grande partie. Nous pouvons notamment voir l’exemple historique de la grande famine suite à l’apparition d’un parasite infectant les pommes de terres, le mildiou, qui a touché l’Europe à la moitié du 19ème siècle, et de manière beaucoup plus ravageuse encore l'Irlande, où l’agriculture reposait essentiellement sur la production de pommes de terre.

Dans le domaine informatique, la monoculture présente des risques notamment sur la rapidité et l’ampleur de l’infection \cite{risksOfMonoculture}. La dépendance à un seul genre de logiciel permet une plus grande facilité de propagation à travers les failles partagés sur les différentes parties d'un système.
Un fait plus récent a attiré notre attention, peu de temps après les attentats qui ont eu lieu à Paris en Janvier 2015, de nombreux sites de municipalités française ont été victime d’attaques. Plusieurs centaines de sites ont été piratés parmi les plus de 35000 communes qui compte la France \cite{communes_INSEE}. A la suite de ces attaques, un audit a été effectué auprès des municipalités \textit{La gazette des communes} . On y apprend qu’environ la moitié des communes disposent de leur propre site avec notamment une très grande biodiversité logicielle, tant au niveau des logiciels que des versions. Ce facteur a-t-il permis de réduire l’ampleur des attaques réussies ?

Nous pouvons voir aujourd’hui que la diversité n’est pas présente au sein des systèmes d’exploitations, où Windows domine le marché avec plus de 90\% des parts, mais aussi au niveau des navigateurs web où son navigateur propriétaire est également prédominant, avec une cependant une tendance à la baisse lors des dernières années \cite{netMarketShare}. 
Dans notre étude, nous nous intéressons à la diversité logicielle, ce que nous définirons par la suite comme l'utilisation d'un ensemble de différents logiciels, mais servant à la même tâche (e.g. navigateur web, serveur web, traitement de texte, ...). Nous définirons alors une entropie des systèmes par rapport aux différentes versions présentes. Celle-ci servira de base pour définir un modèle de propagation d'un \textit{malware} donné dans un système quelconque, en fonction de l'ancienneté moyenne des logiciels qui y sont installés et de leur entropie. Ce modèle prendra également en compte le coût de maintenance du parc informatique pour le défenseur en fonction de l'entropie logicielle de celui-ci, afin d'essayer de trouver le meilleur compromis sur la diversité logicielle.
Afin de valider notre modèle, nous ferons dans un second temps une simulation basée sur un graphe, chaque nœud représentant une machine. Le \textit{malware} essayera ensuite de parcourir le graphe en se propageant par l'intermédiaire d'un lien réseau entre deux machines étant équipées du même logiciel. Nous présenterons ensuite nos résultats, et en tirerons des conclusions sur les situations dans lesquelles il est avantageux pour le défenseur d'investir dans la diversité logicielle de son système afin de le protéger.



%\lipsum[1]
%\hfill August 26, 2015

\section{Revue de littérature}
L'études des vulnérabilités au sein des logicielles n'est pas chose nouvelle. Des études quantitatives cherchant à modéliser le
nombre de failles de sécurités découvertes au cours du temps ont déjà été menées \cite{vulnerabilityDiscovery}. Et ce, avec des
résultats plus que satisfaisants. Il s'agissait dans cet article de parvenir à modéliser le taux de découvertes de vulnérabilités
afin de pouvoir l'anticiper et prévoir à l'avance de consacrer un certain nombre de personnes à la résolution des failles qui
serraient découvertes. Il s'agissait donc plutôt de mieux gérer ses effectifs. Une autre étude, plus centrée sur l'anticipation du
nombre de failles dans un programme, a vu le jour \cite{assessingVulnerabilities}. Cet article introduit également des modèles
permettant de caractériser le nombre de vulnérabilités que l'on trouvera si l'on consacre un certain "effort" à leur recherche.
Par ailleurs, l'existence de malware utilisant plusieurs vulnérabilités pour se propager est déjà connu. On se souviendra ainsi du
vers \textit{Nimda}, qui utilisait 5 vulnérabilités différentes. Du point de vue de la défense, l'idée d'utiliser différents
logicielles, afin de varier les vulnérabilités et donc de limiter les possibilités de l'attaquant n'est pas nouvelle, et plusieurs
travaux ont tenté de caractériser le gain induit par une telle approche. En particulier, \cite{softwareDiversity:Security}
applique la notion d'entropie à des graphes bipartis afin de mesurer la diversité d'un système.
%\lipsum[2]


\section{Méthodologie}
Dans cette section, nous allons présenter les hypothèses que nous avons posé nous présenterons notre analyse du nombre de vulnérabilités commune entre les logiciels.

\subsection{Hypothèses de base}\label{sec:hypothese}
Dans cette  section, nous allons présenter les différentes hypothèses que nous faisons pour définir notre nodèle.
Dans un premier temps, nous parlerons de l'environnement que nous considérons.
Puis dans une deuxième partie, nous discuterons des capacités que possède aussi bien l'attaquant que le défenseur.
Finalement nous poserons les objectifs de chacune des parties.

\subsubsection{Environnement}\label{sec:hypothese:env}
Dans le cadre de cette étude, nous considérons un environnement $E$ composé de $m$ machines utilisant une même application.
Une application représente la fonctionnalité d'un logiciel.
Il y a jusqu'à $n$ logiciel pour une application donnée.
Pour chacun de ces logiciels, nous considérons que le nombre de vulnérabilité augmente avec le temps.
Nous posons également comme hypothèse qu'il n'y a pas de vulnérabilité zero-day exploitables.
Finalement, aucun des logiciels ne possède de vulnérabilité commune.


\subsubsection{Capacité des parties}\label{sec:hypothese:capacite}
Dans le cadre de cette étude et afin de pouvoir étudier l'impact de l'hétérogénéité,  nous considérons que le défenseur ne possède pas de système de protection contre les logiciels malveillant.
Cependant nous supposons qu'il est capable de déployer une mise à jours d'un logiciel sur toutes les machines du système en même temps.

Pour l'attaquant, nous supposons qu'il possède uniquement un malware et que celui-ci n'exploite qu'une seule vulnérabilité.
Le malware ne peut donc cibler qu'un seul logiciel.
Cela nous permet de montrer l'impact maximal sur le système.


\subsubsection{Les objectifs des parties}\label{sec:hypothese:objectifs}
Pour pouvoir étudier l'efficacité du système hétérogène, nous supposons que l'objectif de l'attaquant est de propager un malware et d'infecter le plus de machine possible.
Pour le défenseur l'objectif est d'assurer que le système reste disponible.


%\begin{figure}
%\centering
%%peut être la faire avec des exemple de vulnérabilité réelles
%\begin{tikzpicture}[->,>=stealth',shorten >=1pt,auto,node distance=1cm,
%  thick,version node/.style={circle,fill=blue!15,draw,
%  font=\sffamily\small\bfseries,minimum size=5mm}, vulne node/.style={circle,fill=red!15,draw,
%  font=\sffamily\small\bfseries,minimum size=5mm}]
%  
%  \node[version node] (L0) {L1};
%  \node[version node] (L1) [below of=L0] {L2};
%  \node[version node] (L2) [below of=L1] {L3};
%  \node[version node] (L3) [below of=L2] {L4};
%  \node[version node] (L4) [below of=L3] {L5};
%  
%  \node[vulne node] (V1) [right of=L0, node distance=3cm] {V1};
%  \node[vulne node] (V2) [below of=V1] {V2};
%  \node[vulne node] (V3) [below of=V2] {V3};
%  \node[vulne node] (V4) [below of=V3] {V4};
%  \node[vulne node] (V5) [below of=V4] {V5};
%
%%connexion vulnerabilité 1
%  \draw [-latex'] (V1) -- (L0);
%  \draw [-latex'] (V1) -- (L1);
%  \draw [-latex'] (V1) -- (L2);
%  \draw [-latex'] (V1) -- (L4);
%  
%%connexion vulnerabilité 2
%  \draw [-latex'] (V2) -- (L0);
%  \draw [-latex'] (V2) -- (L1);
%  \draw [-latex'] (V2) -- (L2);
%
%%connexion vulnerabilité 3
%  \draw [-latex'] (V3) -- (L0);
%  \draw [-latex'] (V3) -- (L2);
%
%%connexion vulnerabilité 4
%%  \draw [-latex'] (V4) -- (L4);
%  \draw [-latex'] (V4) -- (L2);
%  \draw [-latex'] (V4) -- (L3);
%
%%connexion vulnerabilité 5
%  \draw [-latex'] (V5) -- (L4);
% 
%  
%\end{tikzpicture}
%\caption{Schéma de la relation entre la version d'un logiciel et les vulnérabilités associées.}
%\label{fig:heteImpactVuln}
%\end{figure}

%Pour analyser cette impact, nous considérons deux éléments, le nombre total $T$ de vulnérabilités du système ainsi que le nombre minimum $M$ de vulnérabilités nécessaires pour contaminer entièrement le système.
%$T$ permet d'estimer la probabilité qu'un attaquant possède un exploit, plus ce nombre est élevé et plus la probabilité est grande.
%$M$ permet d'estimer la complexité pour l'attaquant s'il veut attaquer tout les ordinateurs du système.
%Plus ce nombre est élevé, plus la difficulté est importante.
%
%\paragraph{L'impact des mises à jours}
%Dans le paragraphe précédent, nous avons expliqué l'intérêt qu'il y a à la diversité des versions.
%Toutefois, nous n'avons pas pris en compte l'ancienneté d'une vulnérabilité. 
%L'ancienneté d'une vulnérabilité la rend plus facilement exploitable car plus connue.
%De plus, comme expliqué dans la section~\ref{sec:modelMath}, le fait de faire des mises à jours forcera l'attaquant à augmenter l'effort pour être capable de s'adapter à ce changement.
%
%Pour évaluer l'impact de l'ancienneté d'une version, nous ajoutons une majoration à la probabilité que l'attaquant possède une vulnérabilité.
%En effet, il est peu probable qu'un attaquant possède un exploit pour une vulnérabilité récente alors qu'il est bien plus facile pour lui d'en obtenir un pour une ancienne vulnérabilité.
%Finalement, nous sommes capable d'estimer l'effort nécessaire pour un attaquant pour déterminer les différentes versions de logiciel utilisées lors d'une mise à jour.



%\paragraph{L'impact de l'hétérogénéité du système}
%Le fait d'avoir un système hétérogène amène à une augmentation du nombre total de vulnérabilités.
%Cependant cette diversité peut rendre une attaque globale plus complexe vue que chacun des systèmes sera sensible à des attaques différentes.
%La figure~\ref{fig:heteImpactVuln} exprime cette notion de non concordance entre les vulnérabilités et leur version.
%Dans cette figure, les nœuds "Ln" représentent une version de logiciel et les nœuds "Vn" représentent un numéro de vulnérabilité.
%Nous voyons que certaines versions n'ont aucune vulnérabilité commune avec d'autre tel que "L4" avec et "L5".
%Dans ce cas, il est plus compliqué d'attaquer les deux systèmes simultanément.
%Cependant, si nous prenons le cas de "L1" et "L2", ils sont tout deux sensibles à "V1" et "V2".
%Dans ce cas, la diversité amène une augmentation du risque d'attaques.


\subsection{Analyse du nombre de vulnérabilités par logiciel}
Nous avons extrait depuis la base de données de vulnérabilité du gouvernement américain NVD~\cite{vulnDatabase} le nombre de failles par logiciel.
Cette base de donnée contient une liste des vulnérabilités connues.
La figure~\ref{fig:vulPerLog} montre le nombre de vulnérabilité communes à au moins deux logiciels sur les 70285 vulnérabilités connues.
Il est également nécessaire de considérer que certains logiciels ont des identifications différentes mais sont les même.
Nous avons un total de 4,70\% de logiciels ayant des vulnérabilités communes donc 73\% concerne uniquement 2 logiciels.
Nous pouvons donc considérer que notre hypothèse sur le fait qu'il n'y ait pas de vulnérabilités communes entre plusieurs logiciels est correcte.

\begin{figure}
\includegraphics[width=\linewidth]{data/commonVuln.eps}
\caption{Compte des vulnérabilités communes à au moins 2 logiciels sur les 70285 vulnérabilités connues.}
\label{fig:vulPerLog}
\end{figure}







% An example of a floating figure using the graphicx package.
% Note that \label must occur AFTER (or within) \caption.
% For figures, \caption should occur after the \includegraphics.
% Note that IEEEtran v1.7 and later has special internal code that
% is designed to preserve the operation of \label within \caption
% even when the captionsoff option is in effect. However, because
% of issues like this, it may be the safest practice to put all your
% \label just after \caption rather than within \caption{}.
%
% Reminder: the "draftcls" or "draftclsnofoot", not "draft", class
% option should be used if it is desired that the figures are to be
% displayed while in draft mode.
%
%\begin{figure}[!t]
%\centering
%\includegraphics[width=2.5in]{myfigure}
% where an .eps filename suffix will be assumed under latex, 
% and a .pdf suffix will be assumed for pdflatex; or what has been declared
% via \DeclareGraphicsExtensions.
%\caption{Simulation results for the network.}
%\label{fig_sim}
%\end{figure}

% Note that the IEEE typically puts floats only at the top, even when this
% results in a large percentage of a column being occupied by floats.


% An example of a double column floating figure using two subfigures.
% (The subfig.sty package must be loaded for this to work.)
% The subfigure \label commands are set within each subfloat command,
% and the \label for the overall figure must come after \caption.
% \hfil is used as a separator to get equal spacing.
% Watch out that the combined width of all the subfigures on a 
% line do not exceed the text width or a line break will occur.
%
%\begin{figure*}[!t]
%\centering
%\subfloat[Case I]{\includegraphics[width=2.5in]{box}%
%\label{fig_first_case}}
%\hfil
%\subfloat[Case II]{\includegraphics[width=2.5in]{box}%
%\label{fig_second_case}}
%\caption{Simulation results for the network.}
%\label{fig_sim}
%\end{figure*}
%
% Note that often IEEE papers with subfigures do not employ subfigure
% captions (using the optional argument to \subfloat[]), but instead will
% reference/describe all of them (a), (b), etc., within the main caption.
% Be aware that for subfig.sty to generate the (a), (b), etc., subfigure
% labels, the optional argument to \subfloat must be present. If a
% subcaption is not desired, just leave its contents blank,
% e.g., \subfloat[].


% An example of a floating table. Note that, for IEEE style tables, the
% \caption command should come BEFORE the table and, given that table
% captions serve much like titles, are usually capitalized except for words
% such as a, an, and, as, at, but, by, for, in, nor, of, on, or, the, to
% and up, which are usually not capitalized unless they are the first or
% last word of the caption. Table text will default to \footnotesize as
% the IEEE normally uses this smaller font for tables.
% The \label must come after \caption as always.
%
%\begin{table}[!t]
%% increase table row spacing, adjust to taste
%\renewcommand{\arraystretch}{1.3}
% if using array.sty, it might be a good idea to tweak the value of
% \extrarowheight as needed to properly center the text within the cells
%\caption{An Example of a Table}
%\label{table_example}
%\centering
%% Some packages, such as MDW tools, offer better commands for making tables
%% than the plain LaTeX2e tabular which is used here.
%\begin{tabular}{|c||c|}
%\hline
%One & Two\\
%\hline
%Three & Four\\
%\hline
%\end{tabular}
%\end{table}


% Note that the IEEE does not put floats in the very first column
% - or typically anywhere on the first page for that matter. Also,
% in-text middle ("here") positioning is typically not used, but it
% is allowed and encouraged for Computer Society conferences (but
% not Computer Society journals). Most IEEE journals/conferences use
% top floats exclusively. 
% Note that, LaTeX2e, unlike IEEE journals/conferences, places
% footnotes above bottom floats. This can be corrected via the
% \fnbelowfloat command of the stfloats package.

\section{Conclusion}
Le modèle présenté ici, est, à notre connaissance, le seul modèle permettant de mesurer le nombre d'infection en fonction de la
diversité de de l'ancienneté du système. Bien qu'un travail en amont soit nécessaire, les premiers résultats sont encourageants. 


%\section*{Acknowledgment}
%The authors would like to thank Ericsson for their collaboration on the project.

\bibliographystyle{IEEEtran}
\bibliography{paper1} % IEEEabrv, 
%\printbibliography
\end{document}
