\documentclass[conference]{IEEEtran}
%\documentclass[conference, final]{IEEEtran}

\ifCLASSINFOpdf
  % \usepackage[pdftex]{graphicx}
  % declare the path(s) where your graphic files are
  % \graphicspath{{../pdf/}{../jpeg/}}
  % and their extensions so you won't have to specify these with
  % every instance of \includegraphics
  % \DeclareGraphicsExtensions{.pdf,.jpeg,.png}
\else
  % or other class option (dvipsone, dvipdf, if not using dvips). graphicx
  % will default to the driver specified in the system graphics.cfg if no
  % driver is specified.
  % \usepackage[dvips]{graphicx}
  % declare the path(s) where your graphic files are
  % \graphicspath{{../eps/}}
  % and their extensions so you won't have to specify these with
  % every instance of \includegraphics
  % \DeclareGraphicsExtensions{.eps}
\fi

\usepackage[utf8]{inputenc}
\usepackage{cite}
\usepackage{amsmath}
\usepackage[acronym]{glossaries}
\usepackage{lipsum, adjustbox}

\usepackage{pgf}
\usepackage{tikz}



\usepackage{algorithmic}
\usepackage{array}

\ifCLASSOPTIONcompsoc
  \usepackage[caption=false,font=normalsize,labelfont=sf,textfont=sf]{subfig}
\else
  \usepackage[caption=false,font=footnotesize]{subfig}
\fi

%\usepackage{fixltx2e}
%\usepackage{stfloats}
%\usepackage{dblfloatfix}
\usepackage{url}

\hyphenation{op-tical net-works semi-conduc-tor}


\usepackage[]{todonotes}
%\addbibresource{main.bib}

\begin{document}

\title{Étude de l'influence de la biodiversité logicielle d'un système sur sa probabilité d'infection}

\author{
\IEEEauthorblockN{Philippe Troclet}
\IEEEauthorblockA{
%Email: 
@}
\and

\IEEEauthorblockN{Alexandre Mao}
\IEEEauthorblockA{
%Email: 
@}
\and
\IEEEauthorblockN{Paul Berthier}
\IEEEauthorblockA{
%Email: 
paul.berthier@polymtl.ca}
\and
\IEEEauthorblockN{Thomas Luinaud}
\IEEEauthorblockA{
%Email: 
thomas.luinaud@polymtl.ca}
}

\maketitle

\begin{abstract}

Bien que beaucoup aient mis en avant le risque posé par l'homogénéité au niveau logiciel, on ne sait pas précisément comment la diversité va réellement nuire à l'attaquant. Nous proposons d'apporter des éléments de réponse à l'impact de la diversité sur les attaques en fournissant deux modèles. L'un basé sur l'entropie qui nous permet de connaître le nombre moyen de machines infectées lors d'une attaque en fonction de la configuration logicielle du système, et prenant en compte le surcoût de cette diversité pour le défenseur. Dans le second, les différentes logiciels sont représentés et inter-connectés sous la forme d'un graphe, qui est ensuite parcouru par le \textit{malware.} Nous présentons enfin les résultats de la simulation de nos modèles sous différentes conditions, et montrons que la diversité logicielle permet bien de limiter les étendues des attaques.
\end{abstract} 
%\cite{ArchForCounter}

\IEEEpeerreviewmaketitle

\section{Introduction}
En janvier 2015, à la suite des attentats à Charlie Hebdo, une vague d'attaques a ciblé les sites des municipalités françaises.\cite{courrier}



%\lipsum[1]
%\hfill August 26, 2015

\section{Revue de littérature}
L'études des vulnérabilités au sein des logicielles n'est pas chose nouvelle. Des études quantitatives cherchant à modéliser le
nombre de failles de sécurités découvertes au cours du temps ont déjà été menées \cite{vulnerabilityDiscovery}. Et ce, avec des
résultats plus que satisfaisants. Il s'agissait dans cet article de parvenir à modéliser le taux de découvertes de vulnérabilités
afin de pouvoir l'anticiper et prévoir à l'avance de consacrer un certain nombre de personnes à la résolution des failles qui
serraient découvertes. Il s'agissait donc plutôt de mieux gérer ses effectifs. Une autre étude, plus centrée sur l'anticipation du
nombre de failles dans un programme, a vu le jour \cite{assessingVulnerabilities}. Cet article introduit également des modèles
permettant de caractériser le nombre de vulnérabilités que l'on trouvera si l'on consacre un certain "effort" à leur recherche.
%\lipsum[2]


\section{Méthodologie}
Afin de réaliser notre analyse, nous avons procédé en deux étapes.
Tout d'abord nous avons défini un modèle mathématique permettant d'estimer le coût d'une attaque et de la défense.
Puis nous avons appliqué notre modèle en analysant des jeux de données.
Dans cette section, nous expliquerons tout d'abord notre modèle mathématiques puis nous expliquerons comment nous avons appliqué ce modèle à nos données.
%grandeur concernant la biodiversité du système.

\subsection{Modèle mathématique}\label{sec:modelMath}

Afin de définir notre modèle mathématique, nous avons basé notre approche sur un phénomène physique homologue : la thermodynamique.
Nous allons donc définir une entropie logicielle qui sera liée à l'hétérogénéité du système.
Redéfinissons tout d'abord le premier principe de la thermodynamique~: Il y a conservation de l'énergie. 

\[
\Delta E = \Delta U = W + Q
\]

$\Delta E$ est la variation d'énergie du système, $\Delta U$ est la variation d'énergie interne du système, $W$ est le travail reçu par le système et $Q$ est la chaleur reçue par le système.

La seconde loi de la thermodynamique nous donne :
\[
S_{creation} = \Delta S_{syst} + \Delta S_{ext} \geq 0
\]

Cela se caractérise dans notre cas par un principe simple~: tout changement de configuration dans notre système, quel qu'il soit, crée de l'entropie et donc augmente, au moins temporairement, la difficulté de l'attaque.
Dans le cas ou l'entropie de notre système diminuerai, il y a tout de même une augmentation de l'entropie du côté de l'attaquant. Cela s'explique par le fait que l'état du nouveau système lui est inconnu. Il doit à nouveau refaire toute son analyse avant d'effectuer une nouvelle attaque. On notera également que sans action spécifique du défenseur, en laissant les administrateurs des sous-systèmes faire les mises à jour indépendamment, l'entropie du système global va augmenter de manière naturelle au cours du temps. 

Le second principe peut également s'écrire :

\[
\frac{Q}{T} \leq \Delta S_{syst}
\]

À quoi correspondent toutes ces grandeurs physiques dans notre système ? On va associer Q à l'effort fourni pour effectuer des mises à jours (Si Q est nul, il n'y a pas d'entropie créée), et W sera le travail fourni par l'attaquant pour effectuer son attaque. Du côté du défenseur, Q va en réalité être négatif. En effet, on va associer l'énergie du système à la quantité d'attaques subies. On veut donc en permanence le "refroidir".
De même, la température T va être associée à la vulnérabilité des versions des logiciels du système. Plus les versions sont anciennes, plus elles seront vulnérables et donc la température va augmenter en fonction du temps.
On notera également que plus l'entropie du système est grande, plus le travail à fournir par l'attaquant sera important.



\subsection{Application du modèle}
Dans cette section, nous expliquons comment nous avons appliqué notre modèle mathématiques.
Nous avons dans un premier temps récupérer les données des différentes versions de serveur Web utilisés par les communes françaises ainsi que des vulnérabilités associées.
Finalement nous avons recoupé les différentes information et effectué le calcul mathématiques.

\subsubsection{Récupération des données}
Pour faire une études comparative, nous avons récupéré un jeu de donnée sur les serveurs web utilisés par les communes françaises datant de mars 2015.
Par la suite, nous avons généré un jeu de donnée à partir du même script afin de connaître l'état actuel des systèmes.
Finalement nous avons récupéré les différentes failles de sécurité connu pour les différentes version de ces logiciels.

Les différentes failles de sécurité ont été récupéré depuis~\cite{vulnDatabase}

Une fois les données trouvée, nous avons réalisé un recoupement des données.
Pour cela, nous avons considéré que les versions de logiciel donnés par les serveurs sont les versions réellement utilisés et également que les serveur n'envoyant pas d'informations sont sécurisé de base.


\subsubsection{Application du modèle mathématique}
Notre modèle mathématique prend en compte d'une part la variabilité des différents systèmes dans l'environnement ainsi que les mises à jours.
En effet, nous avons précédemment expliqué qu'une mise à jours amenait un travail supplémentaire à fournir par l'attaquant.
Nous appliquons donc simulation, une où l'on considère la complexité de l'attaque liée à l'hétérogénéité du système et une où l'on considère l'impact des mises à jours sur l'entropie.

\paragraph{L'impact de l'hétérogénéité du système}
Le fait d'avoir un système hétérogène amène à une augmentation du nombre total de vulnérabilité.
Cependant cette diversité peut rend une attaque global plus complexe vue que chacun des systèmes sera sensible à des attaques différentes.
La figure~\ref{fig:heteImpactVuln} exprime cette notion de non concordance entre les vulnérabilités et leur version.
Dans cette figure, les nœuds "Ln" représentent une version de logiciel et les noeuds "Vn" représentent un numéro de vulnérabilité.
Nous voyons que certaine version n'ont aucune vulnérabilité communes avec d'autre tel que "L4" avec et "L5".
Dans ce cas, il est plus compliqué d'attaquer les deux système simultanément.
Cependant, si nous prenons le cas de "L1" et "L2", ils sont tout deux sensibles à "V1" et "V2".
Dans ce cas, la diversité amène une augmentation du risque d'attaques.


\begin{figure}
\centering
%peut être la faire avec des exemple de vulnérabilité réelles
\begin{tikzpicture}[->,>=stealth',shorten >=1pt,auto,node distance=1cm,
  thick,version node/.style={circle,fill=blue!15,draw,
  font=\sffamily\small\bfseries,minimum size=5mm}, vulne node/.style={circle,fill=red!15,draw,
  font=\sffamily\small\bfseries,minimum size=5mm}]
  
  \node[version node] (L0) {L1};
  \node[version node] (L1) [below of=L0] {L2};
  \node[version node] (L2) [below of=L1] {L3};
  \node[version node] (L3) [below of=L2] {L4};
  \node[version node] (L4) [below of=L3] {L5};
  
  \node[vulne node] (V1) [right of=L0, node distance=3cm] {V1};
  \node[vulne node] (V2) [below of=V1] {V2};
  \node[vulne node] (V3) [below of=V2] {V3};
  \node[vulne node] (V4) [below of=V3] {V4};
  \node[vulne node] (V5) [below of=V4] {V5};

%connexion vulnerabilité 1
  \draw [-latex'] (V1) -- (L0);
  \draw [-latex'] (V1) -- (L1);
  \draw [-latex'] (V1) -- (L2);
  \draw [-latex'] (V1) -- (L4);
  
%connexion vulnerabilité 2
  \draw [-latex'] (V2) -- (L0);
  \draw [-latex'] (V2) -- (L1);
  \draw [-latex'] (V2) -- (L2);

%connexion vulnerabilité 3
  \draw [-latex'] (V3) -- (L0);
  \draw [-latex'] (V3) -- (L2);

%connexion vulnerabilité 4
%  \draw [-latex'] (V4) -- (L4);
  \draw [-latex'] (V4) -- (L2);
  \draw [-latex'] (V4) -- (L3);

%connexion vulnerabilité 5
  \draw [-latex'] (V5) -- (L4);
 
  
\end{tikzpicture}
\caption{Schéma de la relation entre la version d'un logiciel et les vulnérabilités associées.}
\label{fig:heteImpactVuln}
\end{figure}

\paragraph{L'impact des mises à jours}
Dans le paragraphe précédent, nous avons expliqué l'intérêt qu'il y a à la diversité des versions.
Toutefois, nous n'avons pas pris en compte l'ancienneté d'une vulnérabilité. 
L'ancienneté d'une vulnérabilité la rend plus facilement exploitable car plus connu.
De plus, comme expliqué dans la section~\ref{sec:modelMath}, le fait de faire des mises à jours forcera l'attaquant à augmenter l'effort pour être capable de s'adapter à ce changement.





% An example of a floating figure using the graphicx package.
% Note that \label must occur AFTER (or within) \caption.
% For figures, \caption should occur after the \includegraphics.
% Note that IEEEtran v1.7 and later has special internal code that
% is designed to preserve the operation of \label within \caption
% even when the captionsoff option is in effect. However, because
% of issues like this, it may be the safest practice to put all your
% \label just after \caption rather than within \caption{}.
%
% Reminder: the "draftcls" or "draftclsnofoot", not "draft", class
% option should be used if it is desired that the figures are to be
% displayed while in draft mode.
%
%\begin{figure}[!t]
%\centering
%\includegraphics[width=2.5in]{myfigure}
% where an .eps filename suffix will be assumed under latex, 
% and a .pdf suffix will be assumed for pdflatex; or what has been declared
% via \DeclareGraphicsExtensions.
%\caption{Simulation results for the network.}
%\label{fig_sim}
%\end{figure}

% Note that the IEEE typically puts floats only at the top, even when this
% results in a large percentage of a column being occupied by floats.


% An example of a double column floating figure using two subfigures.
% (The subfig.sty package must be loaded for this to work.)
% The subfigure \label commands are set within each subfloat command,
% and the \label for the overall figure must come after \caption.
% \hfil is used as a separator to get equal spacing.
% Watch out that the combined width of all the subfigures on a 
% line do not exceed the text width or a line break will occur.
%
%\begin{figure*}[!t]
%\centering
%\subfloat[Case I]{\includegraphics[width=2.5in]{box}%
%\label{fig_first_case}}
%\hfil
%\subfloat[Case II]{\includegraphics[width=2.5in]{box}%
%\label{fig_second_case}}
%\caption{Simulation results for the network.}
%\label{fig_sim}
%\end{figure*}
%
% Note that often IEEE papers with subfigures do not employ subfigure
% captions (using the optional argument to \subfloat[]), but instead will
% reference/describe all of them (a), (b), etc., within the main caption.
% Be aware that for subfig.sty to generate the (a), (b), etc., subfigure
% labels, the optional argument to \subfloat must be present. If a
% subcaption is not desired, just leave its contents blank,
% e.g., \subfloat[].


% An example of a floating table. Note that, for IEEE style tables, the
% \caption command should come BEFORE the table and, given that table
% captions serve much like titles, are usually capitalized except for words
% such as a, an, and, as, at, but, by, for, in, nor, of, on, or, the, to
% and up, which are usually not capitalized unless they are the first or
% last word of the caption. Table text will default to \footnotesize as
% the IEEE normally uses this smaller font for tables.
% The \label must come after \caption as always.
%
%\begin{table}[!t]
%% increase table row spacing, adjust to taste
%\renewcommand{\arraystretch}{1.3}
% if using array.sty, it might be a good idea to tweak the value of
% \extrarowheight as needed to properly center the text within the cells
%\caption{An Example of a Table}
%\label{table_example}
%\centering
%% Some packages, such as MDW tools, offer better commands for making tables
%% than the plain LaTeX2e tabular which is used here.
%\begin{tabular}{|c||c|}
%\hline
%One & Two\\
%\hline
%Three & Four\\
%\hline
%\end{tabular}
%\end{table}


% Note that the IEEE does not put floats in the very first column
% - or typically anywhere on the first page for that matter. Also,
% in-text middle ("here") positioning is typically not used, but it
% is allowed and encouraged for Computer Society conferences (but
% not Computer Society journals). Most IEEE journals/conferences use
% top floats exclusively. 
% Note that, LaTeX2e, unlike IEEE journals/conferences, places
% footnotes above bottom floats. This can be corrected via the
% \fnbelowfloat command of the stfloats package.

\section{Conclusion}
Le modèle présenté ici, est, à notre connaissance, le seul modèle permettant de mesurer le nombre d'infection en fonction de la
diversité de de l'ancienneté du système. Bien qu'un travail en amont soit nécessaire, les premiers résultats sont encourageants. 
\newline

En effet, nos expérimentations montrent, d'une part l'existence d'un lien entre l'entropie et le nombre de machines infectées et d'autre part que l'hypothèse d'une
relation linéaire entre ces quantités est raisonnable. \`A l'heure actuelle, le nombre d'applications ayant suffisamment de
déclinaisons pour que l'entropie d'un système puisse être supérieure à deux est en effet très faible. De ce fait, nos résultats
peuvent s'appliquer à de nombreux cas dans la vie réelle. 
\newline

Nous avons de plus présenté un modèle de coût, qui, une fois les constantes déterminées et une fonction d'évaluation du coût de
maintenance correctement définie, permettra de guider un administrateur réseau dans ses choix pour la maintenance de la
plateforme à sa charge.
\newline

Nous allons clôturer cet article, par le principe suivant: la diversité favorise la sécurité.


%\section*{Acknowledgment}
%The authors would like to thank Ericsson for their collaboration on the project.

\bibliographystyle{IEEEtran}
\bibliography{paper1} % IEEEabrv, 
%\printbibliography
\end{document}
