Afin de réaliser notre analyse, nous avons procédé en deux étapes.
Tout d'abord nous avons défini un modèle mathématique permettant d'estimer le coût d'une attaque et de la défense.
Puis nous avons appliqué notre modèle en analysant des jeux de données.
Dans cette section, nous expliquerons tout d'abord notre modèle mathématiques puis nous expliquerons comment nous avons appliqué ce modèle à nos données.
%grandeur concernant la biodiversité du système.

\subsection{Modèle mathématique}

Afin de définir notre modèle mathématique, nous avons basé notre approche sur un phénomène physique homologue : la thermodynamique.
Nous allons donc définir une entropie logicielle qui sera liée à la "biodiversité" du système.
Redéfinissons tout d'abord le premier principe de la thermodynamique : Il y a conservation de l'énergie. 

\[
\Delta E = \Delta U = W + Q
\]

$\Delta E$ est la variation d'énergie du système, $\Delta U$ est la variation d'énergie interne du système, $W$ est le travail reçu par le système et $Q$ est la chaleur reçue par le système.

La seconde loi de la thermodynamique nous donne :
\[
S_{creation} = \Delta S_{syst} + \Delta S_{ext} \geq 0
\]

Cela se caractérise dans notre cas par un principe simple : tout changement de configuration dans notre système, quel qu'il soit, crée de l'entropie et donc augmente, au moins temporairement, la difficulté de l'attaque.
Dans le cas ou l'entropie de notre système diminuerai, il y a tout de même une augmentation de l'entropie du côté de l'attaquant. Cela s'explique par le fait que l'état du nouveau système lui est inconnu. Il doit à nouveau refaire toute son analyse avant d'effectuer une nouvelle attaque. On notera également que sans action spécifique du défenseur, en laissant les administrateurs des sous-systèmes faire les mises à jour indépendamment, l'entropie du système global va augmenter de manière naturelle au cours du temps. 

Le second principe peut également s'écrire :

\[
\frac{Q}{T} \leq \Delta S_{syst}
\]

À quoi correspondent toutes ces grandeurs physiques dans notre système ? On va associer Q à l'effort fourni pour effectuer des mises à jours (Si Q est nul, il n'y a pas d'entropie créée), et W sera le travail fourni par l'attaquant pour effectuer son attaque. Du côté du défenseur, Q va en réalité être négatif. En effet, on va associer l'énergie du système à la quantité d'attaques subies. On veut donc en permanence le "refroidir".
De même, la température T va être associée à la vulnérabilité des versions des logiciels du système. Plus les versions sont anciennes, plus elles seront vulnérables et donc la température va augmenter en fonction du temps.
On notera également que plus l'entropie du système est grande, plus le travail à fournir par l'attaquant sera important.



\subsection{Application du modèle}
Dans cette section, nous expliquons comment nous avons appliqué notre modèle mathématiques.
Nous avons dans un premier temps récupérer les données des différentes versions de serveur Web utilisés par les communes françaises ainsi que des vulnérabilités associées.
Finalement nous avons recoupé les différentes information et effectué le calcul mathématiques.

\subsubsection{Récupération des données}
Pour faire une études comparative, nous avons récupéré un jeu de donnée sur les serveurs web utilisés par les communes françaises datant de mars 2015.
Par la suite, nous avons généré un jeu de donnée à partir du même script afin de connaître l'état actuel des systèmes.
Finalement nous avons récupéré les différentes failles de sécurité connu pour les différentes version de ces logiciels.

Une fois les données trouvée, nous avons réalisé un recoupement des données.
Pour cela, nous avons considéré que les versions de logiciel donnés par les serveurs sont les versions réellement utilisés et également que les serveur n'envoyant pas d'informations sont sécurisé de base.


\subsubsection{Application du modèle mathématique}


