Afin de réaliser notre analyse, nous avons procédé en deux étapes.
Tout d'abord nous avons défini un modèle mathématique permettant d'estimer le coût d'une attaque et de la défense.
Puis nous avons appliqué notre modèle en analysant des jeux de données.
Dans cette section, nous expliquerons tout d'abord notre modèle mathématiques puis nous expliquerons comment nous avons appliqué ce modèle à nos données.
%grandeur concernant la biodiversité du système.

\subsection{Modèle mathématique}


\subsection{Application du modèle}
Dans cette section, nous expliquons comment nous avons appliqué notre modèle mathématiques.
Nous avons dans un premier temps récupérer les données des différentes versions de serveur Web utilisés par les communes françaises ainsi que des vulnérabilités associées.
Finalement nous avons recoupé les différentes information et effectué le calcul mathématiques.

\subsubsection{Récupération des données}
Pour faire une études comparative, nous avons récupéré un jeu de donnée sur les serveurs web utilisés par les communes françaises datant de mars 2015.
Par la suite, nous avons généré un jeu de donnée à partir du même script afin de connaître l'état actuel des systèmes.
Finalement nous avons récupéré les différentes failles de sécurité connu pour les différentes version de ces logiciels.

Une fois les données trouvée, nous avons réalisé un recoupement des données.
Pour cela, nous avons considéré que les versions de logiciel donnés par les serveurs sont les versions réellement utilisés et également que les serveur n'envoyant pas d'informations sont sécurisé de base.


\subsubsection{Application du modèle mathématique}


