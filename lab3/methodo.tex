



\subsection{Application du modèle}\label{sec:simulation}
Dans cette section, nous expliquons comment nous avons appliqué notre modèle mathématiques.
Nous avons dans un premier temps récupérer les données des différentes versions de serveur Web utilisées par les communes françaises ainsi que des vulnérabilités associées.
Finalement nous avons recoupé les différentes informations et effectué les calculs mathématiques.

\subsubsection{Récupération des données}\label{sec:recupData}
Pour faire une études comparative, nous avons récupéré un jeu de données sur les serveurs web utilisés par les communes françaises datant de mars 2015.
Par la suite, nous avons généré un jeu de données à partir du même script afin de connaître l'état actuel des systèmes.
Finalement nous avons récupéré les différentes failles de sécurité connu pour les différentes version de ces logiciels.

Les différentes failles de sécurité ont été récupéré depuis la base de données de vulnérabilités du gouvernement américain~\cite{vulnDatabase} qui contient des vulnérabilités recensés depuis 2002.
Les informations que nous avons récupéré nous permettent pour chaque vulnérabilité de savoir le logiciel et les différentes versions qui y sont sensibles.


Une fois les données trouvée, nous avons réalisé un recoupement des données.
Pour cela, nous avons considéré que les versions de logiciel donnés par les serveurs sont les versions réellement utilisées et également que les serveurs n'envoyant pas d'informations sont sécurisés de base.


\subsubsection{Application du modèle mathématique}
Notre modèle mathématique prend en compte d'une part la variabilité des différents systèmes dans l'environnement ainsi que les mises à jours.
En effet, nous avons précédemment expliqué qu'une mise à jour amenait un travail supplémentaire à fournir par l'attaquant.
Nous appliquons donc des simulations, une où l'on considère la complexité de l'attaque liée à l'hétérogénéité du système et une où l'on considère l'impact des mises à jour sur l'entropie.

\paragraph{L'impact de l'hétérogénéité du système}
Le fait d'avoir un système hétérogène amène à une augmentation du nombre total de vulnérabilités.
Cependant cette diversité peut rendre une attaque globale plus complexe vue que chacun des systèmes sera sensible à des attaques différentes.
La figure~\ref{fig:heteImpactVuln} exprime cette notion de non concordance entre les vulnérabilités et leur version.
Dans cette figure, les nœuds "Ln" représentent une version de logiciel et les noeuds "Vn" représentent un numéro de vulnérabilité.
Nous voyons que certaines versions n'ont aucune vulnérabilité commune avec d'autre tel que "L4" avec et "L5".
Dans ce cas, il est plus compliqué d'attaquer les deux systèmes simultanément.
Cependant, si nous prenons le cas de "L1" et "L2", ils sont tout deux sensibles à "V1" et "V2".
Dans ce cas, la diversité amène une augmentation du risque d'attaques.


\begin{figure}
\centering
%peut être la faire avec des exemple de vulnérabilité réelles
\begin{tikzpicture}[->,>=stealth',shorten >=1pt,auto,node distance=1cm,
  thick,version node/.style={circle,fill=blue!15,draw,
  font=\sffamily\small\bfseries,minimum size=5mm}, vulne node/.style={circle,fill=red!15,draw,
  font=\sffamily\small\bfseries,minimum size=5mm}]
  
  \node[version node] (L0) {L1};
  \node[version node] (L1) [below of=L0] {L2};
  \node[version node] (L2) [below of=L1] {L3};
  \node[version node] (L3) [below of=L2] {L4};
  \node[version node] (L4) [below of=L3] {L5};
  
  \node[vulne node] (V1) [right of=L0, node distance=3cm] {V1};
  \node[vulne node] (V2) [below of=V1] {V2};
  \node[vulne node] (V3) [below of=V2] {V3};
  \node[vulne node] (V4) [below of=V3] {V4};
  \node[vulne node] (V5) [below of=V4] {V5};

%connexion vulnerabilité 1
  \draw [-latex'] (V1) -- (L0);
  \draw [-latex'] (V1) -- (L1);
  \draw [-latex'] (V1) -- (L2);
  \draw [-latex'] (V1) -- (L4);
  
%connexion vulnerabilité 2
  \draw [-latex'] (V2) -- (L0);
  \draw [-latex'] (V2) -- (L1);
  \draw [-latex'] (V2) -- (L2);

%connexion vulnerabilité 3
  \draw [-latex'] (V3) -- (L0);
  \draw [-latex'] (V3) -- (L2);

%connexion vulnerabilité 4
%  \draw [-latex'] (V4) -- (L4);
  \draw [-latex'] (V4) -- (L2);
  \draw [-latex'] (V4) -- (L3);

%connexion vulnerabilité 5
  \draw [-latex'] (V5) -- (L4);
 
  
\end{tikzpicture}
\caption{Schéma de la relation entre la version d'un logiciel et les vulnérabilités associées.}
\label{fig:heteImpactVuln}
\end{figure}

Pour analyser cette impact, nous considérons deux éléments, le nombre total $T$ de vulnérabilités du système ainsi que le nombre minimum $M$ de vulnérabilités nécessaires pour contaminer entièrement le système.
$T$ permet d'estimer la probabilité qu'un attaquant possède un exploit, plus ce nombre est élevé et plus la probabilité est grande.
$M$ permet d'estimer la complexité pour l'attaquant s'il veut attaquer tout les ordinateurs du système.
Plus ce nombre est élevé, plus la difficulté est importante.

\paragraph{L'impact des mises à jours}
Dans le paragraphe précédent, nous avons expliqué l'intérêt qu'il y a à la diversité des versions.
Toutefois, nous n'avons pas pris en compte l'ancienneté d'une vulnérabilité. 
L'ancienneté d'une vulnérabilité la rend plus facilement exploitable car plus connue.
De plus, comme expliqué dans la section~\ref{sec:modelMath}, le fait de faire des mises à jours forcera l'attaquant à augmenter l'effort pour être capable de s'adapter à ce changement.

Pour évaluer l'impact de l'ancienneté d'une version, nous ajoutons une majoration à la probabilité que l'attaquant possède une vulnérabilité.
En effet, il est peu probable qu'un attaquant possède un exploit pour une vulnérabilité récente alors qu'il est bien plus facile pour lui d'en obtenir un pour une ancienne vulnérabilité.
Finalement, nous sommes capable d'estimer l'effort nécessaire pour un attaquant pour déterminer les différentes versions de logiciel utilisées lors d'une mise à jour.
